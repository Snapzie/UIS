\subsection{Fokuspunkter}
\subsubsection{Et sted med al information (info om sygdom, kost, prøve-svar mm.)}
Alle vores interviewpersoner har det tilfælles, at de har været informationssøgende på egen hånd langt hen af vejen, og vi oplever derfor et stort område, hvor der er plads til forbedring. I tilkobling til minsp.dk og patienters rutinetjek hos fagfolk, bør information vedr. deres individuelle behandling være samlet ét sted.

\subsubsection{En mere personlig Min Sundhedsplatform}
Vores interviewpersoner er bekendte med sygdommen og netværker også med andre diabetikere, blandt andet på de sociale medier. Men de mener, at de i løbet af deres sygdomsforløb har manglet mere personlig kontakt og rådgivning fra fagpersonel. Derfor foreslår vores interviewpersoner, at det bør være muligt at tilbyde patienter og især ny-diagnostiserede patienter personlig kontakt med f.eks. studerende eller sygeplejersker, der kan give almene råd om f.eks. alternativ behandling, støtte og overordnet hjælp til dem som behov.

\subsubsection{Brugervenlighed af Min Sundhedsplatform}
Brugen af Min Sundhedsplatform er varierende blandt vores interviewdeltagere. Nogle benytter flittigt siden, mens andre sjældent eller aldrig benytter den. For at få et indblik i hvor intuitiv og brugervenlig siden er, udførte vi en think aloud test sammen med vores deltagere. Her bad vi dem finde aktuelle diagnoser, journaler, forny deres recept, finde prøvesvar, booke en aftale med egen læge eller hospital, finde et link til 'patienthaandbogen.dk' og finde et link til 'sundhed.dk'. Overordnet fandt alle siden intuitiv og nem at navigere. Jannie som ikke var en hyppig bruger af siden blev begejstret for den information hun kunne finde, men samtidig fandt hun det besværligt at tolke på den information som hun kunne finde. For eksempel finder hun det svært at tolke sine prøvesvar. Ligeledes finder Karen det svært at tolke prøvesvar. Hun fortæller at forskellige sider benytter forskellige enheder til prøver, og at hun selv må omregne sine prøvesvar.\\
Morten og Julia fortæller, at mobilsiden ikke fungerer optimalt, og der kunne gøres store forbedringer der.\\
Overordnet finder vores deltagere siden, uanset om de har benyttet den før eller ej, intuitiv og brugervenlig, og de fleste kunne løse alle vores stillede opgaver.

\subsubsection{Motivation}
En væsentlig og direkte stærk behandling af diabetes kræver ofte en drastisk livsstilsændring, hvilket for mange kan være en stor omvæltning af dagligdagen og dårlige vaner. Dette kan sidestilles med personer som gennemgår et rygestop. Som led i dette kan motivation være en vigtig faktor for at gennemføre sådanne ændringer, hvorfor en både nem og let implementerbar løsning kunne være videomateriale med "gode eksempler" fra folk, der har opnået succes med f.eks alternativ behandling.

\subsubsection{Alternativ behandling}
Ernst følte ikke at han fik tilstrækkelig information omkring alternativer til hans behandling. Efter ikke at have taget sin sygdom synderligt seriøst i en længere periode, fik han et barnebarn og blev klar over, at han var nødt til at lægge sit liv om. Han oplever dog stadig ikke, at den 'medicinske vej' er den rigtige måde at håndtere sin sygdom på. Ernst og Morten er de eneste, som benytter alternativ behandling, hvilket kan skyldes, at diabetikere bliver underinformeret omkring dette, eller at størstedelen af diabetikere foretrækker den kliniske behandling. 
Vores interviewpersoner oplever at en væsentlig og direkte stærk behandling af diabetes ofte kræver en drastisk livsstilsændring, hvilket de mener kan være en stor omvæltning af deres dagligdag og rutinerede vaner. Dette sidestiller Jannie på beskrivende hvis som personer der gennemgår et rygestop. Som led i dette mener flere af vores interviewpersoner at motivation kan være en vigtig faktor for at gennemføre sådanne livstilsændringer. Vores interviewperson 'Ernst' foreslår at en både nem og let implementerbar løsning kunne være videomateriale med "gode eksempler" fra folk der har opnået succes med f.eks alternativ behandling.

\subsubsection{Uniforme prøvesvar}
Flere af vores interviewpersoner oplever, at deres prøvesvar er svære at forstå og at de er i tvivl om betydningen af prøvesvarene. De savner mere uddybende forklaring og mere gense navne og beskrivelser af deres prøvesvar. 
Der var flere af prøvesvarene, som vores interviewpersoner ikke vidste betydningen af alene på baggrund af, at de ikke forstod de lægefaglige udtryk.\\
Nolge af vores interviewpersoner ønsker også, at der stod hvilket hospital, prøvesvarene var blevet taget ved.

\subsubsection{Overflod af fagtermer}
I vores interview med Jannie fik vi hende til at tilgå sine prøvesvar, hvilket hun ikke har gjort før på MSP. Omend hun først var henrykt over at kunne se sine prøvesvar, så blev hun meget skuffet, da hun ikke kunne forstå dem, idet de tekniske detaljer var mange og ikke var præsenteret på en form, som var forståeligt for det almene individ, men i stedet henvendte sig til fagteknisk personale. Færre medicinske betegnelser til fordel for et mere forståeligt sprog ville være at foretrække. 

\subsubsection{Receptfornyelse-påmindelse}
Cirka halvdelen af vores interviewpersoner fortrækker at fornye deres recept via personlig kontakt til egen læge f.eks. via telefon. Den anden halvdelen fornyer deres recept digitalt, men via egen læges interne system eller via sundhed.dk. \\
Det er i dag mulig at lave receptfornyelse via Min Sundhedsplatform ved at sende en mail til hospitalet. Et mål kunne være at flytte måden man receptfornyer over i Min Sundhedsplatform.

\subsubsection{Skabe opmærksomhed omkring Min Sundhedsplatform}
Et gennemgående tema på tværs af vores interviews, har været, at der er manglende kendskab til Min Sundhedsplatform. Selv kronisk syge diabetikere har manglende kendskab til denne side, på trods af at være et stort initiativ fra regionens side til at simplificere kontakt mellem patient og læge. Derfor ville en kampagne for at skabe yderligere kendskab til MSP blandt beboerne i Region Hovedstaden/Sjælland være gavnligt.

\subsubsection{Diskussionsforum for patienter}
I vores samtale med Morten konkluderede vi, at han led et afsavn på diverse diskussionsfora for diabetikere, hvor information omkring diabetes kunne deles blandt de pårørende.
Dog konkluderede vi ligeså i vores samtale med Ernst, at han har en vis skeptisk overfor information, som modtages igennem sådanne fora. En god mellemløsning mellem disse to problemstillinger kunne være at implementere et diskussionsforum på Min Sundhedsplatform, hvor man som patient kan stille og søge på spørgsmål, som læger/sygeplejersker/andre sundhedsfagligt personale kunne besvare, således at der er yderligere tiltro samt kvalitet bag disse svar, som almene patienter har mulighed for at stille. 
