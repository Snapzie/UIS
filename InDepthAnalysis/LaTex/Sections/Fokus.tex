\subsection{Fokuspunkter}
\subsubsection{Et sted med al information (info om sygdom, kost, prøve-svar mm.)}
Alle vores interviewpersoner har det til fælles, at de har været informationssøgende på egen hånd langt hend af vejen, og vi oplever derfor et stort område, hvor der er plads til forbedring. I tilkobling til minsp.dk og patienters rutinetjek hos fagfolk, bør information vedr. deres individuelle behandling være samlet ét sted.

\subsubsection{En mere personlig Min Sundhedsplatform}
Diabetes er en velkendt sygdom, hvor store dele af behandlingen er kendt på forhånd. Derfor bør det være muligt at tilbyde patienter personlig kontakt med f.eks. studerende eller sygeplejesker der kan give almene råd om f.eks. alternativ behandling, støtte og overordnet hjælp til dem som ønsker.

\subsubsection{Brugervenlighed af Min Sundhedsplatform}
Brugen af Min Sundhedsplatform er varierende blandt vores interviewdeltagere. Nogle benytter flittigt siden, mens andre sjældent eller aldrig benytter den. For at få et indblik i hvor intuitiv og brugervenlig siden er, udførte vi en think aloud test sammen med vores deltagere. Her bedte vi dem finde aktuelle diagnoser, journaler, fornye deres recept, prøvesvar, booke en aftale med egen læge eller hospital, finde et link til 'patienthaandbogen.dk' og finde et link til 'sundhed.dk'. Overordnet fandt alle siden intuitiv og nem at navigere. Jannie som ikke var en hyppig bruger af siden blev begejstret for den information hun kunne finde, men samtidig fandt hun det besværligt at tolke på den information som hun kunne finde. For eksempel finder hun det svært at tolke sine prøvesvar. Ligeledes finder Karen det svært at tolke prøvesvar. Hun fortæller at forskellige sider benytter forskellige enheder til prøver, og at hun selv må omregne sine prøvesvar.\\
Morten og Julia fortæller at mobilsiden ikke fungerer optimalt, og der kunne gøres store forbedringer der.\\
Overordnet finder vores deltagere, uanset om de har benyttet siden før eller ej, intuitiv og brugervenlig, og de fleste kunne løse alle vores stillede opgaver.

\subsubsection{Motivation}

\subsubsection{Alternativ behandling}
\subsubsection{Uniforme prøvesvar}

\subsubsection{Overflod af fagtermer}

\subsubsection{Receptfornyelse-påmindelse}

\subsubsection{Skabe opmærksomhed omkring Min Sundhedsplatform}
Et gennemgående tema på tværs af vores interviews, har været, at der er manglende kendskab til Min Sunhedsplatform. Selv kronisk syge diabetikere har manglende kendskab til denne side, på trods af at være et stort initiativ fra regionens side til at simplificere kontakt mellem patient og læge. Derfor ville en kampagne for at skabe yderligere kendskab til MSP blandt beboerne i Region Hovedstaden/Sjælland være gavnligt.

\subsubsection{Diskussionsforum for patienter}
I vores samtale med Morten konkluderede vi, at der var et afsavn på diverse diskussionsfora for diabetikere, hvor information omkring diabetes kunne deles blandt de pårørende.
Dog konkluderede vi ligeså i vores samtale med Ernst, at der er en vis skeptis overfor information, som modtages igennem sådanne fora. En god mellemløsning mellem disse to problemstillinger kunne være at implementere et diskussionsforum på Min Sundhedsplatform, hvor man som patient kan stille og søge på spørgsmål, som læger/sygerplejesker/andre sundhedsfagligt personale kunne besvare, således at der er yderligere tiltro samt kvalitet bag disse svar, som almene patienter har mulighed for at stille. 
