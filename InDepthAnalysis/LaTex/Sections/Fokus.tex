\subsection{Fokuspunkter}
\subsubsection{Et sted med al information (info om sygdom, kost, prøve-svar mm.)}
Alle vores interviewpersoner har det til fælles, at de har været informationssøgende på egen hånd langt hend af vejen, og vi oplever derfor et stort område, hvor der er plads til forbedring. I tilkobling til minsp.dk og patienters rutinetjek hos fagfolk, bør information vedr. deres individuelle behandling være samlet ét sted.
\subsubsection{En mere personlig Min Sundhedsplatform}
Diabetes er en velkendt sygdom, hvor store dele af behandlingen er kendt på forhånd. Derfor bør det være muligt at tilbyde patienter personlig kontakt med f.eks. studerende eller sygeplejesker der kan give almene råd om f.eks. alternativ behandling, støtte og overordnet hjælp til dem som ønsker.
\subsubsection{Brugervenlighed af Min Sundhedsplatform}
\subsubsection{Motivation}
En væsentlig og direkte stærk behandling af diabetes kræver ofte en drastisk livstilsændring, hvilket for mange kan være en stor omvæltning af dagligdagen og dårlige vaner. Dette kan sidestilles med personer som gennemgår et rygestop. Som led i dette kan motivation være en vigtig faktor for at gennemføre sådanne ændringer, hvorfor en både nem og let implementerbar løsning kunne være videomatriale med "gode eksempler" fra folk der har opnået succes med f.eks alternativ behandling.
\subsubsection{Alternativ behandling}
\subsubsection{Uniforme prøvesvar}
\subsubsection{Overflod af fagtermer}
I vores interview med Jannie fik vi hende til at tilgå sine prøvesvar, hvilket hun ikke har gjort før på MSP. Omend hun først var hentrykt over at kunne se sine prøvesvar, så blev hun meget skuffet da hun ikke kunne forstå dem, idet de tekniske detaljer var mange og ikke var præsenteret på en form, som var forståelig for det almene individ men i stedet henvendte sig til fagteknisk personale. Færre medicinske betegnelser til fordel for et mere forståeligt sprog ville være at foretrække. 
\subsubsection{Receptfornyelse-påmindelse}
\subsubsection{Skabe opmærksomhed omkring Min Sundhedsplatform}
Et gennemgående tema på tværs af vores interviews, har været, at der er manglende kendskab til Min Sunhedsplatform. Selv kronisk syge diabetikere har manglende kendskab til denne side, på trods af at være et stort initiativ fra regionens side til at simplificere kontakt mellem patient og læge. Derfor ville en kampagne for at skabe yderligere kendskab til MSP blandt beboerne i Region Hovedstaden/Sjælland være gavnligt.
\subsubsection{Diskussionsforum for patienter}
I vores samtale med Morten konkluderede vi, at der var et afsavn på diverse diskussionsfora for diabetikere, hvor information omkring diabetes kunne deles blandt de pårørende.
Dog konkluderede vi ligeså i vores samtale med Ernst, at der er en vis skeptis overfor information, som modtages igennem sådanne fora. En god mellemløsning mellem disse to problemstillinger kunne være at implementere et diskussionsforum på Min Sundhedsplatform, hvor man som patient kan stille og søge på spørgsmål, som læger/sygerplejesker/andre sundhedsfagligt personale kunne besvare, således at der er yderligere tiltro samt kvalitet bag disse svar, som almene patienter har mulighed for at stille. 
