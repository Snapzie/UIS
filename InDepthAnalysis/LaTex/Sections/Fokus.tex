\subsection{Fokuspunkter}
\subsubsection{Et sted med al information (info om sygdom, kost, prøve-svar mm.)}
Alle vores interviewpersoner har det til fælles, at de har været informationssøgende på egen hånd langt hen af vejen, og vi oplever derfor et stort område, hvor der er plads til forbedring. I tilkobling til minsp.dk og patienters rutinetjek hos fagfolk, bør information vedr. deres individuelle behandling være samlet ét sted.

\subsubsection{En mere personlig Min Sundhedsplatform}
Diabetes er en velkendt sygdom, hvor store dele af behandlingen er kendt på forhånd. Derfor bør det være muligt at tilbyde patienter personlig kontakt med f.eks. studerende eller sygeplejersker der kan give almene råd om f.eks. alternativ behandling, støtte og overordnet hjælp til dem som ønsker.

\subsubsection{Brugervenlighed af Min Sundhedsplatform}
Brugen af Min Sundhedsplatform er varierende blandt vores interviewdeltagere. Nogle benytter flittigt siden, mens andre sjældent eller aldrig benytter den. For at få et indblik i hvor intuitiv og brugervenlig siden er, udførte vi en think aloud test sammen med vores deltagere. Her bedte vi dem finde aktuelle diagnoser, journaler, fornye deres recept, prøvesvar, booke en aftale med egen læge eller hospital, finde et link til 'patienthaandbogen.dk' og finde et link til 'sundhed.dk'. Overordnet fandt alle siden intuitiv og nem at navigere. Jannie som ikke var en hyppig bruger af siden blev begejstret for den information hun kunne finde, men samtidig fandt hun det besværligt at tolke på den information som hun kunne finde. For eksempel finder hun det svært at tolke sine prøvesvar. Ligeledes finder Karen det svært at tolke prøvesvar. Hun fortæller at forskellige sider benytter forskellige enheder til prøver, og at hun selv må omregne sine prøvesvar.\\
Morten og Julia fortæller at mobilsiden ikke fungerer optimalt, og der kunne gøres store forbedringer der.\\
Overordnet finder vores deltagere, uanset om de har benyttet siden før eller ej, intuitiv og brugervenlig, og de fleste kunne løse alle vores stillede opgaver.

\subsubsection{Motivation}
En væsentlig og direkte stærk behandling af diabetes kræver ofte en drastisk livsstilsændring, hvilket for mange kan være en stor omvæltning af dagligdagen og dårlige vaner. Dette kan sidestilles med personer som gennemgår et rygestop. Som led i dette kan motivation være en vigtig faktor for at gennemføre sådanne ændringer, hvorfor en både nem og let implementerbar løsning kunne være videomateriale med "gode eksempler" fra folk der har opnået succes med f.eks alternativ behandling.

\subsubsection{Alternativ behandling}
Ernst følte ikke at han fik tilstrækkelig information omkring alternativer til hans behandling. Efter ikke at have taget sin sygdom synderlig seriøs i en længere periode fik an et barnebarn og blev klar over at er nødt til at lægge sit liv om. Han finder dog stadig ikke at den 'medicinske vej' er den rigtige måde at håndtere sin sygdom. Ernst er den eneste som har fortalt benytter alternativ behandling, hvilket kan skyldes at diabetes patienter bliver underinformeret omkring dette, eller at størstedelen af diabetes patienter foretrækker den kliniske behandling. 

\subsubsection{Uniforme prøvesvar}
Flere af interviews personer oplever at deres prøvesvar er svære at forstå og de er itvivl om betydningen af prøvesvarende. De savner mere uddybende forklaring og mere gense navne til prøvesvarene. 
Der var flere at prøvesvarende interviews personer ikke viste hvad var fordi de ikke forstod de latiske lægefaglige udtryk.\\
Interviews personer ønsker også at der stod hvor / hvilken hospital prøvesvarene var blevet taget.

\subsubsection{Overflod af fagtermer}
I vores interview med Jannie fik vi hende til at tilgå sine prøvesvar, hvilket hun ikke har gjort før på MSP. Omend hun først var henrykt over at kunne se sine prøvesvar, så blev hun meget skuffet da hun ikke kunne forstå dem, idet de tekniske detaljer var mange og ikke var præsenteret på en form, som var forståelig for det almene individ men i stedet henvendte sig til fagteknisk personale. Færre medicinske betegnelser til fordel for et mere forståeligt sprog ville være at foretrække. 

\subsubsection{Receptfornyelse-påmindelse}
Cirka halvdelen af interviews personer fortrækker at fornyer deres recept via persoling kontakt til egen læge f.eks. via telefon. Den anden halvdelen fornyer deres recpet digitalt, men via egen læges interne digitale system eller via sundhed.dk. \\
Det er i dag mulig at lave recpet fornyelse via MinSundhedplatform ved at sende en mail til hospitalet. Et mål kunne være at flytte måden man receptfornyer over i MinSundhedplatform.

\subsubsection{Skabe opmærksomhed omkring Min Sundhedsplatform}
Et gennemgående tema på tværs af vores interviews, har været, at der er manglende kendskab til Min Sundhedsplatform. Selv kronisk syge diabetikere har manglende kendskab til denne side, på trods af at være et stort initiativ fra regionens side til at simplificere kontakt mellem patient og læge. Derfor ville en kampagne for at skabe yderligere kendskab til MSP blandt beboerne i Region Hovedstaden/Sjælland være gavnligt.

\subsubsection{Diskussionsforum for patienter}
I vores samtale med Morten konkluderede vi, at der var et afsavn på diverse diskussionsfora for diabetikere, hvor information omkring diabetes kunne deles blandt de pårørende.
Dog konkluderede vi ligeså i vores samtale med Ernst, at der er en vis skeptisk overfor information, som modtages igennem sådanne fora. En god mellemløsning mellem disse to problemstillinger kunne være at implementere et diskussionsforum på Min Sundhedsplatform, hvor man som patient kan stille og søge på spørgsmål, som læger/sygeplejersker/andre sundhedsfagligt personale kunne besvare, således at der er yderligere tiltro samt kvalitet bag disse svar, som almene patienter har mulighed for at stille. 
