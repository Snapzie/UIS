\subsection{Fokuspunkter}
\subsubsection{Et sted med al information (info om sygdom, kost, prøve-svar mm.)}
\subsubsection{En mere personlig Min Sundhedsplatform}
\subsubsection{Brugervenlighed af Min Sundhedsplatform}
Brugen af Min Sundhedsplatform er varierende blandt vores interviewdeltagere. Nogle benytter flittigt siden, mens andre sjældent eller aldrig benytter den. For at få et indblik i hvor intuitiv og brugervenlig siden er, udførte vi en think aloud test sammen med vores deltagere. Her bedte vi dem finde aktuelle diagnoser, journaler, fornye deres recept, prøvesvar, booke en aftale med egen læge eller hospital, finde et link til 'patienthaandbogen.dk' og finde et link til 'sundhed.dk'. Overordnet fandt alle siden intuitiv og nem at navigere. Jannie som ikke var en hyppig bruger af siden blev begejstret for den information hun kunne finde, men samtidig fandt hun det besværligt at tolke på den information som hun kunne finde. For eksempel finder hun det svært at tolke sine prøvesvar. Ligeledes finder Karen det svært at tolke prøvesvar. Hun fortæller at forskellige sider benytter forskellige enheder til prøver, og at hun selv må omregne sine prøvesvar.\\
Morten og Julia fortæller at mobilsiden ikke fungerer optimalt, og der kunne gøres store forbedringer der.\\
Overordnet finder vores deltagere, uanset om de har benyttet siden før eller ej, intuitiv og brugervenlig, og de fleste kunne løse alle vores stillede opgaver.
\subsubsection{Motivation}
\subsubsection{Alternativ behandling}
\subsubsection{Uniforme prøvesvar}
\subsubsection{Overflod af fagtermer}
\subsubsection{Receptfornyelse-påmindelse}
\subsubsection{Skabe opmærksomhed omkring Min Sundhedsplatform}
\subsubsection{Diskussionsforum for patienter}


