\section{Mål, problemer og behov}
Nedenfor er vores fokuspunkter listet med de mål, behov, problemer og løsninger som vi har identificeret gennem vores analyse af vores interviews.

\begin{tabularx}{\textwidth}{|X|X|X|X|}
	\hline
	\multicolumn{4}{|c|}{\textbf{Et sted med al information (info om sygdom, kost, prøve-svar m.m.}}\\
	\hline
	\textbf{Mål} & \textbf{Problemer} & \textbf{Behov} & \textbf{Løsninger}\\
	\hline
	At diabetes patienter får adgang til behandlingsfremmende oplysninger, som er lægefagligt korrekte og kan hjælpe den enkelte diabetiker med dagligdags udfordringer.&
	Et problem kan være uoverskuelighed og upersonlige oplysninger. Herudover vil et stort problem være behovet og tendensen til at disse informationer ikke bliver benyttet samt at tiltroen til at informationen er korrekt.&
	Efter sammenhold af vores vores interviews vægter vi at behovet for mere information til diabetikere og specielt ny-diagnosticerede opprioriteres.&
	En ekstern side, hvor information kun skrives af fagligt personale for at sikre korrekt infomation, men hvor alle patienter kan stille spørgsmål, idet at nuværende information er mangelfuld.\\
	\hline
	\multicolumn{4}{|c|}{\textbf{En mere personlig Min Sundhedsplatform}}\\
	\hline
	\textbf{Mål} & \textbf{Problemer} & \textbf{Behov} & \textbf{Løsninger}\\
	\hline
	At diabetes patienter får mere personlig opmærksomhed som hjælper den enkelte med blandt andet at opnå større motivation til at behandle sin sygdom.&
	Det kan være en dyr og kompleks løsning at tildele patienter personlig kontakt og vil kræve store omstruktureringer i fagpersonel mv.&
	Behovet er til stede, men bør ikke opprioriteres og anses derfor ikke for værende stort.&
	Dine oplysninger, såsom lidelser, bruges til at skelne mellem nyttig og unyttig information for ikke at overvælde brugeren med information om alverdens sygdomme som ikke er relaterede.\\
	\hline
	\multicolumn{4}{|c|}{\textbf{Brugervenlighed af Min Sundhedsplatform}}\\
	\hline
	\textbf{Mål} & \textbf{Problemer} & \textbf{Behov} & \textbf{Løsninger}\\
	\hline
	At diabetes patienter kan tilgå og forstå de data og oplysninger der samles på platformen.&
	Kan være et problem at tilpasse til forskellige enheder og systemer.
	Det er inkonsistent og ikke intuitivt hvor information ligger på Min Sundhedsplatform.&
	Behovet er ikke stort, men tilstede og der kræves en mindre opmærksomhed på dette plan.&
	Fokus på intuitiv placering af information, dog er siden vist sig at være relativ nem at bruge\\
	\hline
\end{tabularx}

\newpage

\begin{tabularx}{\textwidth}{|X|X|X|X|}
	\hline
	\multicolumn{4}{|c|}{\textbf{Motivation}}\\
	\hline
	\textbf{Mål} & \textbf{Problemer} & \textbf{Behov} & \textbf{Løsninger}\\
	\hline
	Sundhedsfaglig personale og platform motiverer patienter så de får lyst til at foretage en f.eks. omvæltende livsstilsændring&
	Patienter ikke har behov samt lyst til at gennemgå livsstilsændringer og hellere vil fortsætte som de har gjort indtil nu.&
	Behovet er stort for behandlingen af patienterne, men behovet for at minsp.dk kan løfte eller supplere opgaven er lille.&
	Løsningen ligger udenfor MinSP\\
	\hline
	\multicolumn{4}{|c|}{\textbf{Alternativ behandling}}\\
	\hline
	\textbf{Mål} & \textbf{Problemer} & \textbf{Behov} & \textbf{Løsninger}\\
	\hline
	At give diabetespatienter viden og håndgribelige værktøjer til alternativ behandling&
	Der bliver ikke informeret nok omkring alternativ behandling blandt andet at lægefagligt personale fortæller hvorfor, men ikke hvordan alternativ behandling gribes an.&
	Behovet er stort for diabetikere, men forholdsvis lille i forhold til MinSP&
	Løsningen ligger udenfor MinSP\\
	\hline
	\multicolumn{4}{|c|}{\textbf{Uniforme prøvesvar}}\\
	\hline
	\textbf{Mål} & \textbf{Problemer} & \textbf{Behov} & \textbf{Løsninger}\\
	\hline
	At diabetikere får nemmere ved at forstå og bruge de relevante tal som der bliver brugt blandt fagfolk&
	Kan evt. skabe længere arbejdsprocessor hos læger og andre fagfolk, idet de kan behøve længere tid til at fortolke/oversætte diverse resultater.&
	Behovet er stort, da det kan være med til ufyldestgørende behandling&
	MinSP kunne evt. udelukkende tillade én måde at præsentere prøvesvar (f.eks. kun SI-enheder), hvilket kunne bestemmes blandt fagfolk\\
	\hline
	\multicolumn{4}{|c|}{\textbf{Overflod af fagtermer}}\\
	\hline
	\textbf{Mål} & \textbf{Problemer} & \textbf{Behov} & \textbf{Løsninger}\\
	\hline
	Begrænse og overskueliggøre informationer til en bred målgruppe af diabetikere&
	At diabetikere har svært ved at forstå og fortolke egne resultater, hvilket kan forøge behovet for personlig kontakt med lægen for at få oversat nødvendig information&
	Behovet er stort, da informationen, som den præsenteres pt. ikke er forståelig og at det derfor er værdinedsættende for en stor gruppe patienter.&
	Løsningen ligger udenfor MinSP\\
	\hline
\end{tabularx}

\newpage

\begin{tabularx}{\textwidth}{|X|X|X|X|}
	\hline
	\multicolumn{4}{|c|}{\textbf{Receptfornyelse-påmindelse}}\\
	\hline
	\textbf{Mål} & \textbf{Problemer} & \textbf{Behov} & \textbf{Løsninger}\\
	\hline
	Skal minde patienter om at bestille ny medicin i tide således, at de på intet tidspunkt løber tør&
	Pt. er der mange måder at forny recepter/bestille medicin (egen læge pr. telefon/internt system, sundhed.dk, MinSP mm.). Mange ved ikke, at MinSP kan benyttes til at forny medicin og dette vil i værste tilfælde kunne lede til en mangel på kritisk medicin&
	Behovet er stort, da det er en kritisk del af én sammenhængende platform, som vores undersøgelse har vist at være særdeles ønsket. Dog er dette allerede implementeret.&
	En besked i ens E-boks/Email/SMS. Kunne i større grad informere omkring receptfornyelse i MinSP og fjern eksterne muligheder for at fokusere på MinSP\\
	\hline
	\multicolumn{4}{|c|}{\textbf{Skabe opmærksomhed omkring Min Sundhedsplatform}}\\
	\hline
	\textbf{Mål} & \textbf{Problemer} & \textbf{Behov} & \textbf{Løsninger}\\
	\hline
	Målet er at skabe mere opmærksomhed om, ikke blot MinSPs funktionalitet, men også dets eksistens.&
	I vores mange interviews og vores spørgeskema, har vi konkluderet, at selv blandt kronisk syge diabetikere er der et fåtal, som kender til og bruger MinSP. Dette er yderst problematisk hvis man ønsker at lave MinSP om til ét samlet system.&
	Behovet er stort, da en universal platform for information om sygdom og kontakt til hospitaler er værdinedsættende, hvis den brede befolkning ikke kender til platformen eller dens funktionalitet&
	Løsningen ligger udenfor MinSP\\
	\hline
	\multicolumn{4}{|c|}{\textbf{Diskussionsforum for patienter}}\\
	\hline
	\textbf{Mål} & \textbf{Problemer} & \textbf{Behov} & \textbf{Løsninger}\\
	\hline
	At skabe et samlet forum for patienter til udveksling af information og erfaringer&
	Et diskussionsforum hvor alle kan bidrage med erfaringer kan medføre misinformation, idet at ufagligt uddannede patienter ikke er fuldt ud bekendt med sygdomsbehandlingen af diabetes&
	Behovet er ikke synderligt stort som konkluderet af vores spørgeskemaer&
	Ekstern side eller modul til diskussionsforum\\
	\hline
\end{tabularx}
