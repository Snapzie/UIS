\section{Prioriteter}
Ud fra vores analyse af vores fokuspunkter kan vi konkludere  nedenstående prioritering af fokuspunkterne.\\
\begin{enumerate}
	\item \textbf{Samling af al information}. Vi oplevede at alle vores adspurgte deltagere, ved interviewsne, ønskede i større eller mindre grad et samlet sted at finde al sin information, især da de var 'nye i sygdommen'. Det var en generel opfattelse, at patienterne ikke følte, de fik nok information efter og under behandlingen, og de heller ikke fik information om, hvor de ellers kunne finde information. Dette kan også understøttes af vores spørgeskema undersøgelse, hvor der også var stor efterspørgsel efter ét samlet sted at finde information om diabetes.
	\item \textbf{At skabe opmærksomhed omkring Min Sundhedsplatform}. Flere af vores interview deltagere kendte ikke til Min Sundhedsplatform, eller havde et meget begrænset kendskab til siden. Ligeledes i vores spørgeskemaundersøgelse fandt vi at cirka halvdelen, af de adspurgte, ikke kendte siden. Vi finder det derfor meget kritisk at skabe kendskab til siden.
	\item \textbf{Overflod af fagtermer}. 'Overflod af fagtermer' er prioriteret højt, fordi vi oplevede at mange af diabetikerne, vi interviewede oplevede, at de ikke kunne forstå navnet på deres diagnoses eller forstå deres prøvesvar, pga. af de var skrevet med latinske lægefaglige term.
	\item \textbf{Uniforme prøvesvar}. Vi oplevede at vores interviewpersoner fandt det svært, at forstå de forskellige fagtermer som blev benyttet på platformen, og der i høj grad var behov for at give brugeren forståelse for tal og målinger i journaler og prøvesvar.
	\item \textbf{Mere personlig Min Sundhedsplatform}. Flere af vores interviewdeltagere finder Min Sundhedsplatform meget upersonlig, og meget generisk og efterspørger en mere personlig side, hvor der kun er information, som er relevant for dem, og at den er nemt tilgængelig.
	\item \textbf{Alternativ behandling}. Samfundsmæssigt og patientmærssigt er det en fordel for både diabetikeres medicinindtag og deres overordnede helbred at foretage en livsstilsændring - en livsstilsændring som kræver viden om ikke blot hvorfor, men også til hvordan man bør leve anderledes.
	\item \textbf{Motivation}. For at leve med diabetes er det nødvendigt at foretage en livsstilsændring. Flere af vores interviewdeltagere beskriver deres første tid med diabetes som en tid, hvor de levede i en form for fornægtelse af, at de var syge, og de fortsatte derfor deres liv som normalt. 
	Vi finder at ansvaret for motivationen til livsstilsændringen ligger hos det sundhedsfaglige personale, og vi har derfor nedprioriteret dette punkt. Min Sundhedsplatform kan benyttes som supplement til at motivere folk, men 'hoveddelen' af arbejdet bør ligge hos det sundhedsfaglige personale.
	\item \textbf{Receptfornyelse}. Receptfornyelse har lav prioritet, idet denne service allerede findes, men der blot mangler incitament til at benytte den blandt folk, som f.eks. foretrækker personlig kontakt med lægen fremfor MinSP's digitale system.
	\item \textbf{Brugervenlighed af Min Sundhedsplatform}. Efter at have foretaget interviews og think aloud test med vores interview personer, fandt vi ud af at platformen fremkommer intuitiv og der er begrænset behov og få mangler i forhold til at forbedre det nuværende interaktive design, på nær at siden ikke fremkommer særlig mobilvenlig.
	\item \textbf{Diskussionsforum}. I vores interviews og undersøgelser, har vi kunne konkludere, at der er en yderst begrænset interesse for sådanne diskussionsfora, hvilket i høj grad har rod i en mistillid til information, som kommer fra ikke-fagfolk.
\end{enumerate}
