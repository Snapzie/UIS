\section{Prioriteter}
Ud fra vores analyse af vores fokuspunkter er vi kommet frem til nedenstående prioritering af fokuspunkterne.\\
\begin{enumerate}
	\item \textbf{Samling af al information}. Vi oplevede at alle vores adspurgte deltagere ved interviewsne ønskede i større eller mindre grad et samlet sted at finde al sin information, især da de var 'nye i sygdommen'. Det var en generel opfattelse at patienterne ikke følte de fik nok information efter og under behandlingen, og de heller ikke fik information om hvor de ellers kunne finde information. Dette kan også understøttes af vores spørgeskema undersøgelse hvor der også var stor efterspørgsel efter ét samlet sted at finde information om diabetes.
	\item \textbf{At skabe opmærksomhed omkring Min Sundhedsplatform}
	\item \textbf{Overflod af fagtermer}
	\item \textbf{Uniforme prøvesvar}
	\item \textbf{Mere personlig Min Sundhedsplatform}
	\item \textbf{Alternativ behandling}
	\item \textbf{Motivation}
	\item \textbf{Receptfornyelse}
        Receptfornyelse har lav prioritet, idet denne service allerede findes, men der blot mangler incitament til at benytte den blandt folk, som f.eks. foretrækker personlig kontakt med lægen fremfor MinSP's digitale system.
	\item \textbf{Diskussionsforum}
        I vores interviews og undersøgelser, har vi kunne konkludere, at der er en yderst begrænset interesse for sådanne diskussionsfora, hvilket i høj grad har rod i en mistillid til information, som kommer fra ikke-fagfolk.
\end{enumerate}
