\section{Prioriteter}
Ud fra vores analyse af vores fokuspunkter er vi kommet frem til nedenstående prioritering af fokuspunkterne.\\
\begin{enumerate}
	\item \textbf{Samling af al information}. Vi oplevede at alle vores adspurgte deltagere ved interviewsne ønskede i større eller mindre grad et samlet sted at finde al sin information, især da de var 'nye i sygdommen'. Det var en generel opfattelse at patienterne ikke følte de fik nok information efter og under behandlingen, og de heller ikke fik information om hvor de ellers kunne finde information. Dette kan også understøttes af vores spørgeskema undersøgelse hvor der også var stor efterspørgsel efter ét samlet sted at finde information om diabetes.
	\item \textbf{At skabe opmærksomhed omkring Min Sundhedsplatform}. Flere af vores interview deltagere kendte ikke til Min Sundhedsplatform, eller havde et meget begrænset kendskab til siden. Ligeledes i vores spørgeskemaundersøgelse fandt vi at cirka halvdelen af de adspurgte ikke kendte siden. Vi finder det derfor meget kritisk at skabe kendskab til siden.
	\item \textbf{Overflod af fagtermer}
	\item \textbf{Uniforme prøvesvar}
	\item \textbf{Mere personlig Min Sundhedsplatform}. Flere af vores interviewdeltagere finder Min Sundhedsplatform meget upersonlig, og meget generisk og efterspørger en mere personlig side, hvor der kun er information som er relevant for dem, og at den er nemt tilgængelig.
	\item \textbf{Alternativ behandling}
	\item \textbf{Motivation}. For at leve med diabetes er det nødvendigt at foretage en livsstilsændring. Flere af vores interviewdeltagere beskriver deres første tid med diabetes som en tid hvor de levede i en form for fornægtelse af at de var syge, og de fortsatte derfor deres liv som normalt. Vi finder at ansvaret for motivationen til livsstilsændringen ligger hos det sundhedsfaglige personale og vi har derfor nedprioriteret dette punkt. Min Sundhedsplatform kan benyttes som supplement til at motivere folk, men 'hoveddelen' af arbejdet bør ligge hos det sundhedsfaglige personale.
	\item \textbf{Receptfornyelse}
	\item \textbf{Diskussionsforum}
\end{enumerate}