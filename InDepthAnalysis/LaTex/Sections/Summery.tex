\section{Forberedelses fasen}
I forberedelses fasen begyndte vi vores projektarbejde med selv at undersøge hvad diabetes er, og hvad det ville sige at leve med diabetes. Vi begyndte derefter at kigge på hvilke funktioner som min sundhedsplatform havde og hvilke siden manglede, som vi kunne forestille os ville styrke den. Vi kom frem til fire ideer - receptfornyelse, læring og videnscenter, socialt samlingssted og en forberedelse af siden til fremtidens teknologi. Gennem et interview og en spørgeskema undersøgelse kom vi frem til at siden kunne forbedres ved at samle funktioner fra andre af sundhedsvæsenets sider, såsom receptfornyelse, og gøre dem tilgængelige på min sundhedsplatform. Ligeledes fandt vi også frem til at brugerne af siden ofte ikke finder frem til den information de søger, og der kunne derfor også samles mere information omkring diabetes på siden.\\
Nøgleordene for at projektet kan lykkes er brugervenlighed, tilgængelighed og udbredelse af kendskab til siden. Vi fandt at mange af vores adspurgte diabetikere ikke kendte til siden, og dem som gjorde var i mange forskellige aldre. Det er derfor vigtigt for projektet at alle kan benytte de funktioner som vil kunne findes på siden, og at brugervenligheden derfor i høj grad er i fokus.\\
Fra vores strategianalyse fandt vi at det er vigtigt med sikre og intuitive løsninger som brugeren selv kan benytte. Der er behov for et samlet system således at brugeren selv kan finde al den nødvendige information. Samtidigt skal løsningerne være afprøvede og testede i drift, og de skal kunne udvides i fremtiden.
\begin{figure}[H]
	\centering
	\includegraphics[width=\textwidth]{Materials/SeekingInformation}
	\caption{Svarfordeling på spørgsmål om søgning af information.}
\end{figure}
