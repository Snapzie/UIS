\subsection{Interview - Ernst}
Ernst er type-2 diabetiker og arbejder til dagligt som buschauffør. Ernst fik konstateret diabetes tilbage i år 2005, hvor han som bloddoner fik en melding om alt for højt blodtryk. Derfor kom han  til kontrol hos egen læge, hvor han herefter fik sin diagnose.
I starten gik han til egen læge hver 3. måned og fik medformin, som er tablet-medicin for diabetikere. I starten tog Ernst meget let på sin sygdom og de første mange år gik han uden at ændre sin livsstil. Ernst led af stor overvægt og dårlige madvaner – heraf fortalte Ernst at han især var glad for ostemader, hvor 10 stk. dagligt ikke var helt ualmindeligt.
Omkring år 2015 får Ernst beskeden om at han skal være bedstefar, dette var en stor øjenåbner for ham og et opråb om at en livsstilsændring skulle til for at hans levealder og tid sammen med barnetbarnet kunne forlænges. Hans startede derfor hvor det stod værst til – ved hans livstil.
\\ \\
Ernst vælger at tilmelde sig TV2 programmet “Kan man spise sig rask”, hvor han bliver castet, deltager og opnår flotte resultater. Siden har han taget en Umahro-uddannelse (Sundhedsrådgiver) og hjælper ligesindede ved at brede sit budskab vedr. diabetes, og hvordan man kan behandle sin sygdom på andre områder end via de medicinske veje, som han oplever har været omdrejningspunktet i de lægefaglige kræse – hvilket han har en meget stærk holdning til. Herudover er Ernst tilknyttet frivilligcenteret i kommunen, hvor han fortæller og vejleder i KRAM-faktorene (Kost, rygning, alkohol og motion)
\\ \\
Ernst kender ikke til minsundhedsplatform.dk, men er dog tilknyttet Nykøbing Falster Sygehus, hvor han deltager i årlig kontrol. Muligheden for mindst én kontrol årligt som diabetiker blev han først opmærksom på efter, at have været hos egen læge i flere år og har tidligere haft dårlige erfaringer med sin egen læge og dertilhørende henvisninger til informative og støttende foreninger for diabetikere. Det var først da Ernst, fik sit barnebarn at han vendte op og ned på sit liv. 
\\ \\
Ernst er som tidligere nævnt tilknyttet Sygehuset i Nykøbing på Lolland Falster, hvor han årligt går til kontrol. Han mener at fagpersonalet har valgt at sende ham e-mails med de prøvesvar og kontroller han har gået til, men har tilsyndeladende ikke tjekket op på dette et stykke tid, da han ikke rigtigt kan huske det. Han mener at han stadigvæk får e-mails tilsendt, og har hvertfald ikke været opmærksom på minsp.dk. Ernst foretrækker personlig kontakt, hvorfor han i dag også fornyer sin recept via telefon og ikke via lægens interne systemer. Af andre digitale platforme benytter Ernst facebook, hvor han fortæller og rådgiver andre diabetikere med sine egne erfaringer, men er samtidig også skeptik og afholdende fra at modtage gode råd fra disse grupper.
\\ \\
Generelt er Ernst utilfreds med den måde, hvorpå hospitalsvæsenet håndterer diabetes idet, at han får for lidt information omkring alternativ sygdomsbekæmpelse end den medicinske. Han mener at det er økonomiske årsager mellem det offentlige og medicinindustrien der spiller en væsentlig rolle for denne behandlingsmetode. Han fortæller at han var underinformeret omkring blandt andet aflæsning af tal og hvordan han skulle gribe sin livsstilsændring an igennem mange år af sit sygdomsforløb. Ernst savnede især “gode eksempler” hvilket han foreslå for eksempel kunne komme via video på nettet. Overordnet har Ernst manglet motivation og information omkring hvor og hvordan han kunne blive klogere på metoder til at takle sin dagligdag fra den dag han fik konstateret diabetes.

\subsubsection*{Think aloud – minsp.dk} 
Ernst loggede på minsp.dk via hans computer og fandt platformen ret intuitivt. Her kunne Ernst tilgå de forskellige funktioner på siden uden problemer og viste stor interesse og nysgerrighed om denne side han ikke før havde hørt om. Han kiggede på nogle tidligere journaler, men var dog i tvivl om nogle af de aktuelle journalers indhold, om hvorfra og hvilken kontrol disse stammede fra. Han havde ikke yderligere kommentarer til siden som helhed og indtrykket var overordnet positivt.