\section{Interview - Jannie}
Jannie er på førtidspension og har tidligere været pædagog i 8 år. Jannie er normal af bygning, spist almindelig sund kost og altid været sund og rask. Af disse årsager vælger hun i år 2002 at tage forbi Amager Hospital, hvor hun gerne vil melde sig som bloddoner. Dette skal vise sig at være hendes første møde med diabetes på nært hold. Blodprøven viste et højt blodsukkertal og ved senere lægetjek fik hun konstateret diabetes type-2.
\\ \\
I dag tager Jannie medformin, 2 gange om dagen og stikker sig med ”insulin-booster” én gang om ugen, hvilket hun beskriver som værende her, at hun for alvor føler at hun er syg. At stikke sig har altid været ubehageligt og sygdomspåmindende i forhold til medformin som er på tablet-form. Jannie lever som førtidspensionist et meget aktivt liv, hvor hun i ugens løb går til; Yoga mandag, Ipad-kursus tirsdag, synger i gospelkor onsdag, og er frivillig ved hus forbi om torsdagen.
\\ \\
Jannie beskriver sig selv som værende teknisk rutineret, men deltager som førnævnt til Ipad-kursus og vil hele tiden gerne være klogere på den digitale-verden. Heraf blev dette interview også afholdt via facetime som hun nyligt var blevet undervist i. Af digitale platforme med relevans for diabetes har Jannie blandt andet benyttet facebook, men har kort inden interviewet afmeldt sig diverse diabetesgrupper da den løselige information på dette medier ikke sagde hende noget. Herudover benytter hun ”nettet” til informationssøgning – intet specifikt. Jannie har desuden en datter der læser en sundhedsrelevant uddannelse og får derigennem en masse information vedrørende kost og motion.
\\ \\
Jannie kender ikke til minsp.dk, men er tilknyttet diabetesafdelingen på Amager Hospital hvor hun kommer ca. hver 3. måned. Her bliver hun vejet, målt, får taget blodprøver og en gang i mellem deltager i lægekonsultation. Hun går desuden også hos egen læge og bestiller tid og recept til hendes medicin igennem lægens eget interne system.
Som afsluttende del af interviewet spurgte vi Jannie om hun havde en skør idé til en fremtidig digital platform der kunne hjælpe diabetikere. Et forslag til en ren digital løsning havde hun ikke, men havde som diabetiker ønsket at hendes sygdomsforløb havde været grebet anerledes an i starten. Hun ville ønske at der i fremtiden blev sat mere fokus på behandling for nyligt diagnosticerede diabetikere, sådan at de hurtigt og effektivt kunne blive sat i gang med deres behandling og at informationsniveauet både var højere og mindre medicinsk anlagt. Hun foreslår en personlig vejleder kunne være en mulig løsning f.eks. (sygeplejeske) der i starten kunne hjælpe en på rette vej – dette kunne evt. sagtens ske via en platform som minsp.dk og via webcam, foreslår hun.

\subsection*{Think aloud – minsp.dk}
Jannie logger på minsp.dk på sin Ipad og skal først tilmelde sin e-mail da det er første gang hun logger på. Herefter tilgår hun de forkellige funktioner på siden. Hun har aldrig været på siden før og  læser op imens hun navigerer rundt: ”Se din 14 seneste tests” siger hun, og trykker sig ind. Hun bliver henrykt over at kunne se sine resulater fra Amager Hospital og læser videre af en masse latinske lægefaglige udtryk som hun ikke forstår og siger ”Hvordan skal jeg på nogen måde kunne forstå disse ting”. Hun navigerer videre på siden og finder hendes andre diagnoser: diabetes type-2, rygsmerter og diskoskolaps. "Ja det passer jo, det har jeg også", siger hun så. Resten af navigationen rundt på siden sker uden problemer og hun synes at den virker intuitiv og ret brugervenlig. 