\documentclass[english]{article}
\usepackage[utf8]{inputenc}
\usepackage[T1]{fontenc}
\usepackage{babel}
\usepackage{amsmath}
\usepackage{graphicx}
\usepackage{fancyhdr}
\usepackage{listings}
\usepackage{textcomp}
\usepackage{siunitx}
\usepackage{xcolor}
\usepackage{listings}
\definecolor{commentgreen}{RGB}{2,112,10}
\definecolor{eminence}{RGB}{108,48,130}
\definecolor{weborange}{RGB}{255,165,0}
\definecolor{frenchplum}{RGB}{129,20,83}
\lstset {
    language=python,
    frame=tb,
    tabsize=4,
    showstringspaces=false,
    numbers=left,
    upquote=true,
    commentstyle=\color{commentgreen},
    keywordstyle=\color{eminence},
    stringstyle=\color{red},
    basicstyle=\small\ttfamily, % basic font setting
    emph={int,char,double,float,unsigned,void,bool},
    emphstyle={\color{blue}},
    escapechar=\&,
    % keyword highlighting
    classoffset=1, % starting new class
    morekeywords={>,<,.,;,,,-,!,=,~},
    keywordstyle=\color{weborange},
    classoffset=0,
}
\pagestyle{fancy}
\fancyhf{}
\renewcommand{\headrulewidth}{0pt}
\setlength{\headheight}{0pt} 

\begin{document}

\subsection*{Interview - Julia (skrifteligt)}
Har du type 1 eller type 2 diabetes?\\
Type 1 
\\ \\
Bruger  du ’ minsundhedsplatform.dk (msp.dk)’?\\
Ja   
\\ \\
Hvad bruger du msp.dk til?\\
Jeg bruger den ikke helt vildt meget, da jeg også bruger sundhed.dk mest (ifb med recepter og min journal). Men jeg synes den er fin at have som en overbliks-platform i forbindelse med b.la mit sundhedsfaglige team (egen læge, jordemoder og diabeteslæge) samt oversigt over aftaler.
\\ \\
Finder du siden brugervenlig?\\
Den er okay, men i sammenligning med sundhed.dk synes jeg layoutet er klodset. Derudover virker hjemmesiden ikke godt på telefonen (minSP har også en app ved jeg, men den bruger jeg ikke). Derudover synes jeg der er flere problematikker i forbindelse den drop-down menuer (eksempel: hvis jeg skal finde information ang. min journal vil jeg selvfølgelig gå ind under journalnotater, men her får jeg følgende info: dine journalnotater ligger under  dine tidligere aftaler – personligt synes jeg ikke det giver ”brugervenligt” mening)
\\ \\
Savner du generelt information for diabetikere på msp.dk? (f.eks. om kost vejledning, motion m.v.)\\
Både ja og nej – hvis jeg brugte det mere kunne der måske være fedt at have alle oplysninger et sted. Men grundet min minimale brug er det ikke noget jeg synes jeg mangler.
\\ \\
Hvor finder du ellers generel information omkring diabetes?\\
Diabetesforeningens egen hjemmeside. 
\\ \\
Forstår du dine prøvesvar? Eller savner du uddybende forklaring til dine prøvesvar?\\
             F.eks.:  \\
             Information om hvorfor de er taget?  \\
             Information om hvor de er taget (hospital)?\\
Jeg savner helt klart mere information. Både hvor de er taget men også de mere ”gense” navne. Jeg ved fx godt at glukose betyder langtidsblodsukker, men har også andre prøvesvar hvor jeg ikke aner hvad det er, fordi navnet kun er skrevet på ”læge-sprog”.
\\ \\
Får du information omkring hvilken medicin du skal tage og hvornår?\\
Nej, det er noget jeg aftaler med min læge på Steno diabetes center
\\ \\
Får du information omkring hvor din diagnose er stillet?\\
Nej
\\ \\
Er din diagnose tilstrækkelig forklaret?\\
Nej
\\ \\
Bruger du siden (platformen) til at forny din recept?\\
Nej
\\ \\
Benytter du meddelelses-funktionaliteten til at kunne kommunikere med hospitalet?\\
Jeg får en besked til påmindelse om mine tider, men ellers bruger jeg det ikke
\\ \\
Savner du at kunne kommunikere med egen læge via. msp.dk?\\
Nej
\\ \\
Hvis du har flere diagnoser: Er dine prøvesvar, aftaler, information mm., for de forskellige diagnoser, tilstrækkeligt adskilt?\\
NA
\\ \\
Benytter du msp app’en?\\
Nej
\\ \\
Hvis du prøver at logge ind på ’ minsundhedsplatform.dk’ (med nem.id). Finder du det så nemt at:\\
(Svar: ’Ja’ eller ’Nej’ - og uddyb gerne)
\\ \\¨
Finde: Din kontaktlæge på hospitalet?\\
Ja
\\ \\
Finde: Aktuelle diagnoser?\\
Nej – intet sted står der fx at jeg har type 1 diabetes 
\\ \\
Finde: Journal fra sidste kontrol?\\
Nej
\\ \\
Finde: Hvordan du fornyer en recept?\\
NA
\\ \\
Finde: Prøvesvar?\\
Ja
\\ \\
Finde: Hvordan du booker en aftale?\\
Ja
\\ \\
Finde: Information omkring din sygdom / link til ’patienthaandbogen.dk’?\\
Nej
\\ \\
Finde: Link til ’sundhed.dk’?\\
Nej
\\ \\
Har du ideer / ønsker til, hvad en fremtidig version af msp.dk skal indeholde for diabetikere?\\
Under forsiden eller profil kunne der godt være en mere personlig drop down menu med fokus på type 1 diabetes (eller andre personlige diagnoser for den sags skyld). 




\end{document}