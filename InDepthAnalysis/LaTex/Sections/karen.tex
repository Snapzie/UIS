\subsection{Interview - Karen}
Karen er pensioneret sekretær og fik konstateret type-2 diabetes i en ret ung alder. I starten fortalte hun at hun var meget underinformeret, hvor hun var bange for sin sygdoms konsekvenser og ikke vidste hvordan hun skulle håndtere sin sygdom. Dette fik hende til at spise utrolige mængder rugbrød uden pålæg som et desperat forsøg på at komme hendes sygdom til livs, hvilket resulterede i at hun tog meget på i vægt i den periode. Hun mente at informationsniveauet i starten var utilstrækkelig og manglede generelt information til hvordan hun skulle gribe sin sygdom an i forhold til fremtidig bahandling. I dag tager hun medformin, 2 piller morgen og aften, og lever et godt og aktivt liv trods sin sygdom.
\\ \\
Karen har tidligere været tilknyttet Bispebjerg sygehus, hvor hun blandt andet har deltaget i projekter og forløb vedrørende hendes sygdom. Hvordan hun i sin tid fik svar og resultater fra disse kontroller og forsøg kunne hun ikke huske, da det var lang tid siden. Karen er i dag ikke tilknyttet et sygehus, men kender dog til minsp.dk, hvor hun en enkelt gang eller to har læst prøvesvar vedr. Hendes sukkertal, som hun mener hendes læge indtaster, men det var hun desværre ikke klar over. 
\\ \\
Hun har generelt ikke benyttet minsp.dk meget, men kun været nysgerrig på hvad det var. Hun har herudover også downloaded app'en, som hun synes er meget brugervenlig i udseende, men bruger den ikke overhovedet. Hun finder minsp.dk brugervenlig og kan i det hele taget godt lide at benytte digitale platforme, hvilket også betyder hun fornyer recept hos egen læge digitalt, via læges eget interne system. Hun går til kontrol ved egen læge hver 3 mdr. Hun synes at det er ægerligt at der ikke findes et samlingssted for diabetikere online, hvor mange ting er samlet under ét hvor man kan få den fornødne information man har behov for. En kommentar vedr. mållinger var at der bliver benyttet forskellige talbetegnelser for blodsukkeret, blandt fagfolk, digitale målemaskiner o.lign, hvilket hun fandt ret frustrerende, og at det var hende selv som skulle finde information og omregne tal. 
Som en afsluttende del af interviewet bad vi Karen foreslå en skør idé til et digitalt produkt/platform som kunne hjælpe diabetikere i hendes øjne. Karens forslag var et mere specificeret samlingssted for diabetikere, hvor der var mere håndgribelig information til især ny-diagnosticerede patienter.
\\ \\
Karen sidder tilligemed i en styregruppe, hvor hun som brugerrepræsentant for diabetikere kommer med indput til omlægning eller tilknytning af anden telefonlinje til diabetikere istedet for 1813, så man akut kan få hjælp af fagspecialister i diabetes 24 timer i døgnet.

\subsubsection*{Think aloud - minsp.dk}
Karen har benyttet minsp.dk til at tjekke prøvesvar og blodsukkertal før, men er ikke jævnlig bruger af platformen. Hun finder siden brugervenlig og har benyttet den under dette interview, men var generelt meget kortfattet omkring hendes oplevelser under brugen af siden. Under think-aloud-testen navigerede Karen rundt på siden, hvor hun gennemgik forskellige funktionaliteter på hjemmesiden under vores forespørgsel, heriblandt: Kontakt til hospital/læge, aktuelle diagnoser og prøvetal, journal samt information vedr. Sygdom. Dette klarede Karen uden nogle problemer og hun virkede generelt til at være en rutineret bruger af digitale platforme og fandt hurtigt frem til de punkter som fremlagt under denne test. En bemærkelse var dog at Karen under testen havde lidt problemer med at tilgå "receptfornyelse samt booking/aftaler" som lå under samme menupunkter, om det skyldes et internetproblem eller selve siden er uvist, men hun virkede til at være teknisk i stand til at finde disse punkter. 
\\ \\
Overordnet set var hendes opfattelse af siden var ret positiv og at den fremkom brugervenlig og at det generelt var et overskueligt system set fra et interaktivt synspunkt, men syntes at det var svært at forstå tal og prøvesvar generelt og kunne godt have brug for bedre vejledning til hendes tidligere prøvesvar.
