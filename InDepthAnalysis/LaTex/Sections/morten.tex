\documentclass[english]{article}
\usepackage[utf8]{inputenc}
\usepackage[T1]{fontenc}
\usepackage{babel}
\usepackage{amsmath}
\usepackage{graphicx}
\usepackage{fancyhdr}
\usepackage{listings}
\usepackage{textcomp}
\usepackage{siunitx}
\usepackage{xcolor}
\usepackage{listings}
\definecolor{commentgreen}{RGB}{2,112,10}
\definecolor{eminence}{RGB}{108,48,130}
\definecolor{weborange}{RGB}{255,165,0}
\definecolor{frenchplum}{RGB}{129,20,83}
\lstset {
    language=python,
    frame=tb,
    tabsize=4,
    showstringspaces=false,
    numbers=left,
    upquote=true,
    commentstyle=\color{commentgreen},
    keywordstyle=\color{eminence},
    stringstyle=\color{red},
    basicstyle=\small\ttfamily, % basic font setting
    emph={int,char,double,float,unsigned,void,bool},
    emphstyle={\color{blue}},
    escapechar=\&,
    % keyword highlighting
    classoffset=1, % starting new class
    morekeywords={>,<,.,;,,,-,!,=,~},
    keywordstyle=\color{weborange},
    classoffset=0,
}
\pagestyle{fancy}
\fancyhf{}
\renewcommand{\headrulewidth}{0pt}
\setlength{\headheight}{0pt} 

\begin{document}

\section*{Interview - Morten}
Morten er 41 år og fik for ca. 8-10 år siden konstateret diabetes type-2. Morten havde gået syg i en længere periode (konstant været snottet, følt sig skidt tilpas og tabt sig mange kg.). Derfor tager han til sin egen læge. Hos lægen får Morten konstateret diabetes type-2 og målt et alt for højt blodsukkerniveau.
\\ \\
Morten er tilknyttet egen læge, hvor han går til personlig konsultation og får målt sit blodsukker. Han går hos sin læge ca. hver 3 måned og fornyer sin recept via lægens interne system som han finder nemt og ligetil, og han foretrækker elektronisk bestilling. Morten er ikke tilknyttet et hospital, men er dog bekendt med minsp.dk, hvor han går ind og tjekker sine blodsukkertal, vitaminmangel, kolesteroltaltal og andre målinger som hans egen læge har indtastet. Morten kender ikke til app'en, men er positiv over at sådan en findes og vil efter egen udsagn formentligt benytte platformen oftere end hjemmesiden. 
\\ \\
Morten synes ikke han mangler noget vedrørende hans sygdom. Hans kone er sygeplejeske og han får derigennem en masse information vedrørende diabetes. Han fortæller dog at han i starten af sin sygdom manglede en masse information og at det først var for nyligt at han han havde fået et gennembrud med en alternativ behandlingsform end den medicinske vej. Ifølge ham selv er der alt for lidt fokus på alternativ behandling og han har derfor i samråd med hans kone fulgt et low-carp kost-program og omlagt sine kostvaner drastisk, hvilket har resulteret i, at han i dag har lagt ”medformin” på hylden som han ellers har taget fast i ca. 9 år. Det betyder at Morten i dag ikke tager medicin overhovedet – med stor succes. Han forslår et digitalt samlingssted for diabetikere, hvor man kan få rådgivning om selv de mindste kostråd, så folk kommer væk fra ”løs” informationssøgning på facebook og diverse internet-sider online. 

\subsection*{Think aloud – minsp.dk}
Morten kender godt til minsp.dk og benytter den en gang imellem, men ikke mere end de besøg han har haft hos lægen. Morten var kort af ord og vi prøvede ikke at tilgå siden sammen, men han kommenterede på sine oplevelser af siden alligevel. Han fandt ikke siden besværlig at navigere rundt på, men kommenterede at siden hvertfald ikke var mobiloptimeret, hvilket han synes være lidt irriterende når han skulle bruge den. De resterende funktionaliteter som forefindes på hjemmesiden benytter han ikke.













\end{document}