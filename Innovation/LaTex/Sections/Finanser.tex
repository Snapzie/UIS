\section{Økonomi}
At implementere vores visioner til en fyldestgørende grad ville være en dyr affære med behov for programmører og andre designere, som regionen og/eller staten skal finansiere. Evt. ville jobbet skulle outsources til et firma, som f.eks. det amerikanske Epic, der oprindeligt implementerede Min Sundhedsplatform. Hvorvidt investeringen er omkostningen værd, har så rod i, om implementationen giver en tilstrækkelig forbedret brugeroplevelse, og om pengene kan spares andre steder. Eksempelvis vil vores vision om forbedret receptfornyelse kunne spare penge på sekretærer, som ville behøve at besvare færre emails. Samtidigt ville det tillade de dyre, veluddannede læger at fokusere mere på det arbejde, udelukkende de kan udføre, fremfor at bruge tiden på mere administrativt arbejde. Det samme gælder for vores vision om uniforme prøvesvar, da dette ville bidrage til, at læger og lægesktretærer ikke ville behøve at bruge tid på at forklare patienters målinger. Dette er essentielt at medbringe i overvejelserne, når man skal beslutte at lave disse tilføjelser eller ej. 
