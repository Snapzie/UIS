\section{Økonomi}
At implementere vores visioner til en fyldestgørende grad ville være en dyr affære med behov for programmører og andre designere, som regionen og/eller staten skal finansiere. Programmører er dyre i drift, og at implementere eksempelvis receptfornyelse ville kræve meget kodearbejde, da dette system skal kunne "snakke" med lægernes interne systemer, af hvilke der er mange. Evt. ville jobbet skulle outsources til et firma, som f.eks. det amerikanske Epic, der oprindeligt implementerede Min Sundhedsplatform. I begyndelsen af projektet, ville det kræve en investering i at påbegynde implementationen, samtidigt med, at læger og sekretærer ender med at skulle lave mere administrativt arbejde end førhen (Da de skal komme med input til implementationen), hvilket vil betyde en dobbeltudgift under udarbejdelse af projektet. Når projektet er færdiggjort, forudser vi dog ikke mange løbende udgifter. Dette skyldes primært, at implementationen ikke medfører, at det sundhedsfaglige personale skal oplæres yderligere i, hvordan man bruger systemet, da det ikke ændrer deres arbejdspraksis. Hvorvidt investeringen er omkostningen værd, har så rod i, om implementationen giver en tilstrækkelig forbedret brugeroplevelse, og om pengene kan spares andre steder. Eksempelvis vil vores vision om forbedret receptfornyelse kunne spare penge på sekretærer, som ville behøveskulle besvare færre emails. Samtidigt ville det tillade de dyre, veluddannede læger at fokusere mere på det arbejde, udelukkende de kan udføre. Det samme gælder for vores vision om uniforme prøvesvar, da dette ville bidrage til, at læger og lægesekretærer ikke ville behøve at bruge tid på at forklare patienters målinger. Dette er essentielt at medbringe i overvejelserne, når man skal beslutte at lave disse tilføjelser eller ej. 
