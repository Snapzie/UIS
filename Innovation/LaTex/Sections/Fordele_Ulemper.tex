% Konsekekvens-analyse resultatet giver Fordele og Ulemper
%s. 207 Konsekvansanalyse: 'gennemgå systematisk de forslåene visioner'
%
%  Teknik: Virtuelle kort, evt. Brug SWOT s. 269-270
%
% Pricippet Samlet Vision - s. 72 trekant:  IT-udvikling / kvalifikationsudvkling / Organisatoriske udvikling : 
%
%Diskution af
% hvad er ændringen
% hvordan pårvirker det mennesker der skal bruge det
% Arbejds Organiserings ændringer / skaber det ændringer i arbejdsgange
% Forankring ('knytter sig til' personalet og organiserings)
% krav til uddanelse /kvalifikations krav
% Økonomi
% /Matcher det med Region H. forretnings og IT stategi
\subsection{Fordele og Ulemper}
%Virtuelle kort
\subsubsection{Region Sjællands forretnings- og it-strategi}
\textbf{Fokuspunkt 1, Receptfornyelse}\\
Fordelen ved at implementere en automatisk receptfornyelses funktionalitet er at brugerene kan betjene sig selv og vil opleve en bedre service. Det vil muligvis også reducere sundhedspersonalets ressourceforbrug. Dette opfylder målsætninger i Region Sjællands it-strategi.\\
Det vil også opfylde Regions Sjællands it-infrastruktur om at MinSP bliver en samlet side for alle funktioner.\\
Ulempen er hvis funktionaliteten 'Receptfornyelse' ikke bliver brugt af patienterne, men at patienterne stadigvæk bruger sundhed.dk, da vi vudere at funktionaliteten vil være rimelig dyr at implementere og det vil være i modstrid med Regions Sjællands it-strategi om at "Regionen vil have fuldt udbyde af sine investeringer, og de nye løsninger skal bruges i deres fulde omfang" og vil give en dårlig cost/benefit for regionen.\\ 
Så vil en bedre løsning være bare at oprette den nye hovedmenu  'Receptfornyelse' hvor man herunder kan link til sundhed.dk. Det er dog i modstrid med Region Sjællands it-infrastruktur, om at "Brugeren skal kunne finde løsninger i samme brugervendte arbejdsgange, uden at brugerne skal opleve behov for at navigere mellem forskellige løsninger". \\
\textbf{Fokuspunkt 2, Samling af al information}
 % !  PUNKT 8 - hente information fra andre systemer - i IT-stategien skal også med her
En 'standardløsning' med link sider med information om diabetes stemmer overens med Region Sjællands forretnings- og it-strategi. 
Som er at patienter i så høj grad så mulig skal kunne betjene sig selv samtidig med at de for en bedre service grad og regionen reducere sit ressourceforbrug.\\
%
\textbf{Fokuspunkt 3, Uniforme Prøvesvar}\\
Forbedret løsningen til visning af prøvesvar vil giver patienterne bedre service som understøtter it-strategien.\\
% Er løsningen dyr at kode??
Det er det sundhedsfagligepersonale der skal lave de danske beskriver af de forskellige typer prøvesvar, dette matcher ikke med Region Sjællands ønske om at reducere sit ressourceforbrug. Så det vil være en dyr løsning\\
%
% Kunder og levandørere: de skal sige god for til links i 'Samling af al information'
\subsubsection{Relationer mellem sundhedspersonalet og mellem afdelinger}