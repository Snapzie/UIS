% Konsekekvens-analyse resultatet giver Fordele og Ulemper
% ud fra Scenarios
%s. 207 Konsekvansanalyse: 'gennemgå systematisk de forslåene visioner'
%
%  Teknik: Virtuelle kort, evt. Brug SWOT s. 269-270, Design skitsere
%
% Pricippet Samlet Vision - s. 72 trekant:  IT-udvikling / kvalifikationsudvkling / Organisatoriske udvikling : 
%
%Diskution af
% hvad er ændringen
% hvordan pårvirker det mennesker der skal bruge det
% Arbejds Organiserings ændringer / skaber det ændringer i arbejdsgange
% Forankring ('knytter sig til' personalet og organiserings)
% krav til uddanelse /kvalifikations krav
% Økonomi
% /Matcher det med Region H. forretnings og IT stategi
\section{Konsekvensanalyse: Fordele og Ulemper}
I dette afsnit ser vi på hvorfor visionerne er relevante for Region Sjællands forretnings- og it-strategi, for sundhedspersonalet og for patienter og lervandører. \\
Til hjælp i vores vurdering af fordele og ulemper af visionerne om 'Receptfornyelse' og 'Samling af al information' på MinSP, har vi udarbejdet to scenarier, hvor visionerne ses ud fra henholdsvis en bruger og en sundhedsfagligs synspunkt. Scenarierne ses nedenfor under \textcolor{red}{punkt 6}.
\todo{Virtuelle kort / SWOT analyse}
\subsection{Region Sjællands forretnings- og it-strategi}
\textbf{Prioritet 1, Receptfornyelse}\\
Fordelen ved at implementere en automatisk receptfornyelses funktionalitet er, at brugerne kan betjene sig selv og vil opleve en bedre service. Det vil muligvis også reducere sundhedspersonalets ressourceforbrug. Dette opfylder målsætninger i Region Sjællands it-strategi \footnote{Projektgrundlag, s. 6, Afsnit 3.1 IT strategien, punkt 7 }.\\
Det vil også opfylde Region Sjællands it-strategi om infrastruktur, at MinSP bliver en samlet side for alle funktioner og har integrerede selvbetjenings løsninger. \footnote{Projektgrundlag, s. 7, Afsnit 3.4 It-infrastruktur punkt 3 og 4}\\
Ulempen er, hvis funktionaliteten 'Receptfornyelse' ikke bliver brugt af patienterne, men at patienterne i stedet stadigvæk bruger sundhed.dk, fordi vi vuderer, at funktionaliteten vil være rimelig dyr at implementere, og det vil være i modstrid med Region Sjællands it-strategi om at "Regionen vil have fuldt udbytte af sine investeringer, og de nye løsninger skal bruges i deres fulde omfang" \footnote{Projektgrundlag, s. 5, Afsnit 3.1 IT strategien, punkt 2}. Det vil sige, at det så vil være en dårlig cost/benefit for regionen.\\ 
En bedre løsning ville så være kun at oprette den nye hovedmenu 'Receptfornyelse' og herunder linket til sundhed.dk. Det vil dog være i modstrid med Region Sjællands it-infrastruktur, om at "Brugeren skal kunne finde løsninger i samme brugervendte arbejdsgange, uden at brugerne skal opleve behov for at navigere mellem forskellige løsninger" \footnote{Projektgrundlag, s. 7, Afsnit 3.4, It-infrastruktur, punkt 3}. 
\\\\
\textbf{Prioritet 2, Samling af al information}\\
En 'standardløsning' med link til sider med information om diabetes stemmer overens med Region Sjællands forretnings- og it-strategi om, at patienter i så høj grad som muligt skal kunne betjene sig selv samtidig med, at de får en bedre servicegrad og regionen reducerer sit ressourceforbrug. \footnote{Projektgrundlag, s. 6, Afsnit 3.1 IT strategien, punkt 7} \\
Det er også en prioritet i it-strategien, at systemer skal kunne hente information fra andre systemer, så dette understøttes også af vores løsning. \footnote{Projektgrundlag, s. 6, Afsnit 3.1 IT strategien, punkt 8} \\
Løsningen opfylder også en målsætning i regionens it-infrastruktur om at: "Regionen ønsker systemer, som kan hente information fra andre systemer, således at brugeren kun skal henvende sig et enkelt sted".\footnote{Projektgrundlag, s. 7, Afsnit 3.4 It-infrastruktur, punkt 3}
\\\\
\textbf{Prioritet 3, Uniforme Prøvesvar}\\ % ? - Er løsningen dyr at kode
En forbedret løsning til visning af prøvesvar vil give patienterne bedre service som er i overensstemmelse med regions it-strategi. \footnote{Projektgrundlag, s. 6, Afsnit 3.1 IT strategien, punkt 7}\\
Det er det sundhedsfaglige personale, der skal lave de danske beskrivelser for de forskellige typer af prøvesvar. Dette matcher ikke med Region Sjællands ønske om at reducere sit ressourceforbrug \footnote{Projektgrundlag, s. 6, Afsnit 3.1 IT strategien, punkt 7}, så cost/benefit skal vurderes i forhold til det forbedret service niveau for patienterne. 
\subsection{Relationer mellem sundhedspersonalet og mellem afdelinger}
\textbf{Prioritet 1, Receptfornyelse}\\
Det er en fordel for læge-sekretærerne, at de ikke længere skal huske at svare tilbage til patienter omkring receptfornyelse og skal bruge tid på det.
% Ulemper -?
\\\\
\textbf{Prioritet 2, Samling af al information}\\
Det er en fordel for sundhedspersonalet, at patienterne i højere grad selv kan søge information, da det vil betyder mindre arbejde for sundhedspersonalet.\\
Det vil give en bedre kommunikation mellem sundhedspersonalet og patienterne, når patienter, via informationslinksene, er bedre informeret om deres sygdom, behandlingsforløb, kost og motion m.v.
% Ulemper -?
\\\\
\textbf{Prioritet 3, Uniforme Prøvesvar}\\
Prøvesvarene bliver mere forståelige for patienterne, og det vil lette arbejdet for sundhedspersonalet med at besvare spørgsmål og forklare prøvesvar for patienterne.\\
Det er en ulempe, at sundhedspersonalet får mere arbejde med at indrapportere ekstra data. \\ Det er også sundhedspersonalet, der skal lave de danske tekster.
\subsection{Patienter og lervandørere}
\textbf{Prioritet 1, Receptfornyelse}\\
Patienter kan bestille receptfornyelse det samme sted, som hvor de kan se prøvesvar og kommunikere med hospitalet, og de kan følge med i, hvor langt deres bestilling er nået.\\
Det kommer muligvis til at gå hurtigere for patienten at få fornyet sin recept, fordi der ikke skal skrives beskeder til hospitalet. Patienten undgår arbejdet med at skrive beskeder og at skulle huske, hvad medicinen hedder.\\
Patienten får også et medicin kort, der er nemt at finde og adgang til en historik over, hvad de har fået af medicin gennem tiden i deres sygdoms forløb.\\
Der tages hensyn til brugervenligheden ved at gøre receptfornyelses modulet synligt.\\ 
Der er ingen ulemper for patienten, for de vil stadigvæk kunne bruge sundhed.dk eller kunne kontakte egen læge, hvis de hellere vil det.\\
Apoteket vil muligvis komme til at skulle tjekke to steder for medicin bestilling: sunhed.dk og MinSP.
\\\\
\textbf{Prioritet 2, Samling af al information}\\
Ifølge vores interview-undersøgelse har det at kunne søge information og vejledning om sin sygdom været meget efterspurgt. Så løsningen vil opfylde et stort behov hos brugerne. \\
Linket til patienthåndbogen er gjort mere synlig ved at være flyttet fra "Historik" under "Sundhedsdata" til "Information om dine diagnoser" under "Profil". Det har gjort informationssøgning på MinSP mere brugervenlig for patienterne.\\
Vi har udvidet med to ekstra informationslink, hvor patienterne kan finde information om nogen af de forhold de har efterspurgt, f.eks. video om hvordan man måler blodglukose, kost-vejledning mm.\\
Det er nemt at udvide modulet med flere informationssider, hvis dette ønskes af patienterne.\\
Der skal muligvis skaffes tilladelse fra leverandørerne af linksene til at bruge linksene.
% Patienter og levandørere: de skal sige god for til links i 'Samling af al information'
\\\\
\textbf{Prioritet 3, Uniforme Prøvesvar}\\
Patienterne vil bedre kunne forstå, hvad et prøvesvar omhandler, når der er tilknyttet en standard beskrivelse på dansk. Patienten vil også kunne se, hvor deres prøver er taget.\\ Begge tilføjelser er, jf. vores interviews, et ønske fra patienterne.\\
Ved at samle prøvesvarene pr. diagnose giver det patienterne et bedre overblik over deres prøvesvar.