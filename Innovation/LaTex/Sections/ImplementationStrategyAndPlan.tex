\subsection{Implementations-strategi og plan}
For at kunne skabe en overordnet strategi for implementationen af vores visioner, er det essentielt først at genkende forskellige aspekter ved processen:
\begin{itemize}
  \item Der skal bygges på et allerede eksisterende domæne (Min Sundhedsplatform), så det er vigtigt at evaluere hvorledes implementationen af vores visioner ville passe ind
    i dette domæne, samt at vores egen implementation ikke skal være til ulempe for det allerede ekstisterende domæne. 
  \item Implementationen af vores visioner sker af et meget lille hold (fire mennesker), som er en hindring for omfanget af vores projekt. At evaluere et realistisk ambitionsniveau er derfor kritisk for at være i stand til at lave en brugbar implementation. 
  \item I sammenhæng med ovensstående, er det også vigtigt at evaluere hvilke værktøjer, som bruges til at implementere vores visioner, da der ikke er tale om et firma med stor tidligere erfaring. Derfor er det vigtigt at holde implementationen simpel.
  \item Tidsbegrænsningerne skal tages højde for, således at der benyttes en implementationsstrategi som er fornuftig i forhold til den mængde tid, der stilles til rådighed.
  \item Risiko-faktoren for vores implementation skal vurderes således, at der benyttes en fornuftig fremgangsmåde for at implementere projektet. Idet vores projekt til dels tager udgangspunkt i at streamline proessen, hvormed man bestiller medicin, er det essentielt, at systemet er brugbart og intuitivt. Derfor konkluderes projektet at have en høj risiko-faktor, som medfører, at vi i vores realiseringsprocess må finde en måde at forsiikre projektets kvalitet på.
\end{itemize}

I dette konkrete tilfælde konkluderes det udfra ovenstående, at en inkremental strategi bør benyttes, hvor dele af projektet gradvis laves og hver del evalueres inden arbejde på næste begynder, idet dette bidrager til at skabe et vel-implementeret projekt, hvor hver del passer sammen. Denne tidskrævende men effektive fremgangsmåde er mulig, da projektet har et mindre omfang (er en prototype) således, at vi kan tillade os denne tidskrævende fremgangsmåde.\\
Til selve implementationen passer værktøjer som HTML/SQL/CSS rigtig godt, da de i høj grad muliggøre en interaktiv prototype til at illustrere vores visioner, samt at de er anvendelige for vores mindre hold af folk uden megen efaring indenfor producering af software.


