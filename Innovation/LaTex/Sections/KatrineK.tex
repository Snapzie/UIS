\textbf{Katrine Bundgaard Kaalund, AC-medarbejder i Data og Digitaliseringsenheden,\\
	Sundhedsplatforms teamet, 
	Nordsjællands Hospital, Kvalitets – og udviklingsafdelingen}\\\\
\textit{Katrine Bundgaard Kaalund, AC-medarbejder fra Sundhedsplatforms teamet på NOH, fortæller om arbejdsgangen i forbindelse med receptfornyelse som receptfornyelse fungere idag via MinSP:}\\\\
Kære Sarah.\\
Jeg hjælper dig gerne med at få besvaret dine spørgsmål.\\\\
På NOH (Nordsjællands Hospital) er vi ved at implementere en ny arbejdsgang (på nogle afdelinger) ift. håndtering af recepter, som kommer som forespørgsler fra patienten enten fra MinSP eller over telefon.\\\\
\textbf{Den nye arbejdsgang:} \\
Når patienten skriver til afdelingen via MinSP kan patienten markere at det drejer sig om en receptfornyelse. Denne besked ender hos sekretæren som kan gå ind og forberede recepten. Herefter sender sekretæren anmodningen videre til en læge som blot skal ind og godkende recepten. \\
Fordelen - er at sekretæren eller en sygeplejerske kan forberede det hele til lægen. Derudover får sekretæren en automatisk meddelelse tilbage når lægen har godkendt/afvist recepten.\\\\
\textbf{Den nuværende arbejdsgang:}\\
Patienten skriver til sekretæren og anmoder om en recept, sekretæren sender en besked til lægen om at pågældende patient ønsker recept på XX. Herefter går længen ind og finder medicinen, laver en recept og godkender den. Sekretæren får ikke her en automatisk besked tilbage og skal derfor selv sørge for at følge op med lægen efterfølgende.\\\\
Jeg ved ikke så meget om den knap der er i Sundhed.dk, men så vidt jeg ved, skal alle recepter igennem en godkendelsesprocedure. Dvs. vi undgår nok ikke at der skal en læge indover for at godkende recepten. \\
Men hvis denne recept-knap med det sammen kunne forberede hele recepten automatisk ville den sparre sekretæren/sygeplejersken for nogle klik og involvering i arbejdsgangen. \\
På den anden side bør man overveje fordele ved at der er en sekretær/sygeplejerske indover først. En fordel kan være at sekretæren/sygeplejersken kan skrive tilbage til patienten at nu er deres recept godkendt, men at de næste gang har mulighed for at skrive til deres praktiserende læge for at få lavet en recept, hvis det er et lægemiddel der tillader det. Det er nemlig muligt for patienterne at skrive til afdelingen lang tid efter at patientens forløb er afsluttet i afdelingen.
\\\\
Det er lægerne på hospitalet der godkender recepterne. På hospitalerne bliver recepterne lavet i SP. Praktiserende læger kan ikke gå i SP.\\\\
Med venlig hilsen\\
Katrine Bundgaard Kaalund
