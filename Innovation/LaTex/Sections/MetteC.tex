\textbf{Mette Christensen, MinSP projektkonsulent, \\
	Region Sjælland, Sundhedsstrategisk Planlægning}\\\\
\textit{Mette Christensen, MinSP projektkonsulent, fortæller vedr. tilføjelse af receptfornyelses modul på MinSP at:}\\\\
Kære Sarah.\\
Det lyder spændende med jeres projekt.\\\\
Et receptmodul vil være en hjælp. Det er en smart måde at gøre det på – der er bare så meget der skal udvikles og tages stilling til. Alt medicin burde jo komme fra FMK og vi har ikke integration til FMK i MinSP. Derfor viser vi slet ikke medicin.
\\
Vi har overvejet om man på nogen måde ville kunne vise den medicin der kom fra SP – uden at vise hvor meget medicin man skulle tage – men blot navnet på den medicin man skal bestille på sygehuset, for at patienten har en knap de kan trykke på når de vil bestille mere.
\\
Hvis vi viser alt medicin, vil listen ikke være komplet før vi kan integrere til FMK. Så det vil være uhensigtsmæssigt at patienten kan se en medicin liste i MinSP som kan være uaktuel. Hvis nu patienten efter sin udskrivelse går til egen læge, og han laver om på det – vil medicin i MinSP ikke være korrekt.
\\
Det er lidt tricky. Vi overvejede om vi skulle udvikle en funktion der ”bare viser” selve lægemidlet, uden dosis og antal, men vi er ikke kommet videre mere det.  
\\
Meget medicin får patienten jo via egen læge – og de bruger ikke MinSP, så det vil kun være sygehus medicin man skal kunne bestille.
\\\\
Medicinkortet på sundhed.dk sender jo recepten til egen læge og der er kun egen læge at vælge. I MinSP er det også lidt sværere, idet du får en liste af afdelinger – og hvis vi har en recept knap – hvor går recepten så til? Skal systemet bagom regne ud hvem der har givet recepten første gang? Eller skal patienten selv vælge et sted de vil sende den hen?\\
Når patienten i dag, sender en besked eller ringer ind efter en recept, sender der besked til lægen på vagt som så udfylder recepten elektronisk. Det er aldrig egen læge idet egen læge ikke har adgang til SP, så sekretæren/spl. bruger tid på at undersøge hvor recepten hører til og hvem der skal have den tilsendt – så den rigtige læge kan udfylde det.\\\\
Så der er mange forhold der skal tænkes over, og vi har ikke kunne løse det endnu.\\\\
Venlig hilsen\\
Mette Christensen


