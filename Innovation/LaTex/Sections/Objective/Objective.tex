\section{Målsætninger}
\subsection{Målsætning og præmis for design af projektet}
Målsætningen er at realisere vores vision om en forbedret version af Min Sundhedslpatform. Dette indebærer at lave en prototype, som bedst muligt løser de problemstillinger, vi nåede frem til i vores dybdeanalyse. Mere specifikt ønsker vi at skabe en prototype, som illustrerer hvordan en implementation af receptfornyelse på Min Sundhedsplatform kunne fungere. Når prototypen er lavet, vil det blive nemmere at vurdere, hvorvidt vores problemstillinger, som vi kom frem til i dybdeanalysen, faktisk er blevet løst og/eller om nye problemstillinger opstår som konsekvens af implementationen. 
\subsection{Hovedpointer fra inline analysis}
Under inline-analysen lavede vi en overordnet analyse af, hvad Min Sundhedsplatform er og hvad diabetes er, med forhåbningen om at identificere områder, hvormed Min Sundhedsplatform kunne yde en bedre service for diabetikere. Vi konkluderede følgende:
\subsubsection{Interessenter}
Vi konkluderede følgende interessenter:
\begin{itemize}
\item Diabetikere, som vil skulle benytte implementationen af vores visioner.
\item Region Hovedstaden og Region Sjælland, da disse er områderne hvor Min Sundhedsplatform benyttes.
\item Sundhedsministeren, hvis implementationen skulle spredes til at gælde hele landets borgere ffremfor kun dem i Region Hovedstaden/Sjælland.
\end{itemize}
Igennem et spørgeskema med diabetikere, konkluderede vi ligeså, at der var generel interesse i at muliggøre receptfornyelse igennem Min Sundhedsplatform. Ligeså konkluderede vi, at diabetikere i søg på information ofte tyede til enten lægen eller Google. Et godt alternativ til dette ville være at publicere relevant info på Min Sundhedsplatform, som er skrevet af sundhedsfagligt personale og som man derfor har større tilid til som diabetiker, end noget man har fundet et andet sted på nettet. 
\subsection{Hovedpointer fra dybdeanalysen}
I dybdeanalysen, med formål yderligere at specificere problemstillingen, vi ønsker at løse, udarbejdede vi nogle interviews med diabetikere. Under disse interviews gennemgik vi deres forhold til Min Sundhedsplatform og fik dem til at benytte den. Sidstnævnte gav et indblik i, hvor brugervenlig siden er.\\
Udfra disse interviews og spørgeskemaerne fra vores inline-analyse konkluderede vi nogle fokuspunkter, som indeholder, men ikke er begrænset til:
\begin{itemize}
  Samling af information på Min Sundhedsplatform
  Personlig-gørelse af min Sundhedsplatform, så den bedre henvender sig til lden konkrete bruger.
  Receptfornyelse på Min Sundhedsplatform
\end{itemize}
Af ovenstående fokuspunkter er særligt det om samling af information og receptfornyelse på Min Sundhedsplatform blevet arbejdet med yderligere i denne fase og udgør fundament for vores prototype. 
