\section{Formål}
\subsection{Forundersøgelsens formål og udgangspunkt}
Målsætningen er at realisere vores vision om en forbedret version af Min Sundhedsplatform. Dette indebærer at lave en prototype, som bedst muligt løser de problemstillinger, vi nåede frem til i vores dybdeanalyse. Mere specifikt ønsker vi at skabe en prototype, som illustrerer, hvordan en implementation af receptfornyelse på Min Sundhedsplatform kunne fungere. Når prototypen er lavet, vil det blive nemmere at vurdere, hvorvidt vores problemstillinger, som vi kom frem til i fordybelsesfasen, faktisk er blevet løst og/eller om nye problemstillinger opstår som konsekvens af implementationen.\\
De visioner, som vi har implementeret i vores prototype, er blevet understøttet af vores forundersøgelse. Med dette menes, at vi har dannet grundlag for, hvorfor vores visioner bør implementeres og hvordan, de skal implementeres for bedst muligt at leve op til vores visioner. Desforuden er der under forundersøgelsen udarbejdet en strategi for, hvordan implementationen af vores visioner kunne foregå, for at gøre implementationen bedst og billigst muligt. 
\subsection{Hovedpunkter fra fokuseringsfasen}
Under fokuseringsfasen lavede vi en overordnet analyse af, hvad Min Sundhedsplatform er, og hvad diabetes er, med forhåbningen om at identificere områder, hvormed Min Sundhedsplatform kunne yde en bedre service for diabetikere. Vi konkluderede følgende:
\subsubsection{Interessenter}
Af interessenter identificerede vi følgende:
\begin{itemize}
\item Diabetikere, som vil skulle benytte implementationen af vores visioner.
\item Region Hovedstaden og Region Sjælland, da disse er områderne hvor Min Sundhedsplatform benyttes.
\item Sundhedsministeren, hvis implementationen skulle spredes til at gælde hele landets borgere fremfor kun dem i Region Hovedstaden/Sjælland.
\end{itemize}
Igennem et spørgeskema med diabetikere, konkluderede vi, at der var generel interesse i at muliggøre receptfornyelse igennem Min Sundhedsplatform.\\
Desuden fandt vi, at diabetikere i søg på information ofte tyede til enten lægen eller Google. Et godt alternativ til dette ville være at publicere relevant info på Min Sundhedsplatform, som er skrevet af sundhedsfagligt personale og som man derfor har større tilid til som diabetiker, end noget man har fundet et andet sted på nettet. 
\subsection{Hovedpunkter fra fordybelsesfasen}
I fordybelsesfasen, hvis formål er yderligere at specificere problemstillingen, vi ønsker at løse, udarbejdede vi nogle interviews med diabetikere. Under disse interviews gennemgik vi deres forhold til Min Sundhedsplatform og fik dem til at benytte den. Sidstnævnte gav et indblik i, hvor brugervenlig siden er.\\
Ud fra disse interviews, samt spørgeskemaerne fra fokuseringsfasen, konkluderede vi nogle fokuspunkter, som indeholder, men ikke er begrænset til:
\begin{itemize}
  \item Samling af information på Min Sundhedsplatform
  \item Personliggørelse af Min Sundhedsplatform, så den bedre henvender sig til den konkrete bruger.
  \item Uniforme prøvesvar, der er let forståelige for den almene borger.
\end{itemize}
I fordybelsesfasen konkluderede vi, at der ikke var behov for receptfornyelse via Min Sundhedsplatform, hvilket vi siden hen har revurderet. Vi har derfor, på trods af vores konklusioner under fordybelsesfasen, valgt at gå videre med receptfornyelse som grundlag for vores prototype. 

