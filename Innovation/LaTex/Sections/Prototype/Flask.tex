\subsection{Prototype Flask}
Vores Flask applikation tager udgangspunkt i \textit{'prototype/\_\_init\_\_.py'} hvor oplysningerne om vores database er sat på linje 8. Det er også i denne fil vores blueprint 'Main' bliver registreret. Vi har også defineret en commandline command 'test' som kører vores tests.\\
I filen \textit{'prototype/models.py'} er der lavet klasser over de af vores tabeller som vi benytter i vores prototype. Dette gør at vi kan skrive queries som henter en række ned fra vores database og lave et objekt over oplysningerne. Det er også i denne fil vi har defineret vores queries som metoder.\\
I filen \textit{'prototype/Main/routes.py'} finder vi vores routes som definerer de sider vores prototype benytter. De mest interessante er \textit{'/receptfornyelse'}, \textit{'/renew/<med\_name>/<med\_conc>'} og \textit{'/login'}. Disse sider benytter før nævnte objekter og queries til at udføre vores business logic før siderne bliver renderet. Vores 'receptfornyelse' route kører vores \textit{'get\_Active\_Prescriptions'} query for at finde brugerens aktive recepter. Resultatet af denne query er en liste af 'Prescription' objekter som bliver givet videre til vores view der itererer over alle objekterne i listen for at skabe en dynamisk webside. Vores '/renew/<med\_name>/<med\_conc>' route bliver besøgt fra vores receptfornyelses side når brugeren klikker for at fornye sin recept. Denne route sørger for at den tidligere aktive recept bliver gjort inaktiv, og at der bliver lavet en ny recept som bliver den aktive.\\
I filen \textit{'prototype/tests/test\_model.py'} findes vores tests som kan køres som beskrevet nedenfor.
