\subsection{Prototype model}
Ud fra vores vision om et receptfornyelsesmodul, vores udarbejdede user stories og vores MoSCoW prioriteringer har vi udviklet nedenstående ER-diagram over vores domæne. 
\todo{
	% Sarah: Jeg er ikke færdig med review, men har lige dette punkt:
	Sarah: Jeg er usikker på om man kan lave det den vej - Har du undersøgt det? 
	Jeg lavede det modsat -  lave current user stories ud fra ER-diagrammet (som beskrevet i opgavenformuleringen man skulle) 
	- og brugte interviews til furture user stories 
	- og MoSCoW til priortering af visioner 
	- og priorteringen bruges til valg af prototype.
}
\\
\missingfigure{ER diagram}\\
Vores recept entitet kan unikt identificeres ud fra vores andre entiteters nøgler, og er derfor lavet en weak entity. Blandt vores ønskede funktioner er historik og medicinkort. Da der i en log over historik godt kan fremstå den samme medicin til den samme patient flere gange, har det været nødvendigt at tilføje en 'Historik' entitet som udover ovennævnte kan holde tidspunktet for fornyelsen af recepten.\\
Medicinkortet kan 'laves' ud fra recept entiteten givet hvilken patient der ønsker at se sit medicinkort. Det har derfor ikke været nødvendigt at lave en entitet til denne feature.\\

Vi har herefter omsat vores entiteter til tabeller på samme hvis som beskrevet i \textit{'Database Systems. The Complete Book'}\footnote{Prentice Hall, Database Systems. The Complete Book, s. 157-163}. Mange af vores entiteter kan laves direkte til en tabeller som ses nedenfor:
\begin{align*}
	&\textrm{Patient}(\textrm{\underline{CPR}, Password})\\
	&\textrm{Apotek}(\textrm{\underline{Navn}, Addresse})\\
	&\textrm{Medicin}(\textrm{\underline{Navn}, \underline{Styrke}})\\
\end{align*}
Ligeledes kan 'Diagnose' laves direkte. Denne tabel vil komme til at indeholde alle de diagnoser som kan gives, og er nødvendig da en patient kan have mere end en diagnose.\todo{High potential for change}
\begin{equation*}
\textrm{Diagnose}(\textrm{\underline{Sygdom}})
\end{equation*}
Vi har herefter vores 'Recept' entitet. Denne er weak, og henter nøgler fra de fleste andre entiteter. 'Status' attributten vil blive benyttet i overensstemmelse med vores vision om at patienten selv skal kunne følge med i receptfornyelsesprocessen. 'Udstedt' attributten vil blive benyttet af medicinkortet, og vil aldrig ændre sig da den repræsenterer hvornår patienter første gang blev ordineret med medicinen. 'Udløber' attributten vil blive benyttet til at sende reminders til patienterne.\todo{For detaljeret? Nogen grund til at forklare disse attributter?} 
\begin{equation*}
	\textrm{Recept}(\textrm{\underline{ApoNavn}, \underline{MedID}, \underline{PatCPR}, \underline{fornyet}, status, udstedt, udløber})
\end{equation*}
Hvis vi skulle lave vores relationer til om til tabeller ville der blive en en masse redundancy da alle på nær en enkelt relation vil allerede være subsets af andre tabeller. Derfor er den eneste relation som bliver lavet om til en tabel 'Behandles for'. Her vil begge nøgler være foreign keys til henholdsvis 'Patient' og 'Dignose'.
\begin{equation*}
\textrm{BehandlesFor}(\textrm{\underline{PatCPR}, \underline{Sygdom}})
\end{equation*}

\subsubsection{Known shortcommings}
Det er muligt at fornye den samme recept på forskellige apoteker, givet tuplen (\underline{ApoNavn}, \underline{MedID}, \underline{PatCPR}, \underline{fornyet}) vil være unik når apoteket ændres.