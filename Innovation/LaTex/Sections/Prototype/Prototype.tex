\section{Prototype}
I de følgende afsnit vil vores prototype blive gennemgået. Først gennemgås vores ER-diagram og hvordan vi er kommet frem til vores endelige tabeller i databasen. Deri er en overordnet gennemgang af hvordan entiteterne i ER-diagrammet hænger sammen med vores forbedringsforslag om receptfornyelse, og hvilke features vores prototype understøtter. Dernæst gennemgås hvordan vores flask applikation er sat op, hvordan den køres og hvordan man kører vores tests. Efterfølgende er en showcasing af prototypen og de understøttede features, og afslutningsvis har vi en liste af known shortcommings.\\
Det har været vores mål, at prototypen skulle basere sig på, og efterligne vores mock-ups. Men bemærk, at vores prototype på visse punkter æstetisk adskiller sig fra vores Mock-up, men at de stadigvæk opfylder samme egenskaber og behov. Det har desværre ikke været helt muligt inden for vores tidsramme at opfylde alle punkter indenfor visionen. 

\subsection{Prototype model}
Ud fra vores vision om et receptfornyelsesmodul, vores udarbejdede user stories og vores MoSCoW prioriteringer har vi udviklet nedenstående ER-diagram over vores domæne. 
\todo{
	% Sarah: Jeg er ikke færdig med review, men har lige dette punkt:
	Sarah: Jeg er usikker på om man kan lave det den vej - Har du undersøgt det? 
	Jeg lavede det modsat -  lave current user stories ud fra ER-diagrammet (som beskrevet i opgavenformuleringen man skulle) 
	- og brugte interviews til furture user stories 
	- og MoSCoW til priortering af visioner 
	- og priorteringen bruges til valg af prototype.
}
\\
\missingfigure{ER diagram}\\
Vores recept entitet kan unikt identificeres ud fra vores andre entiteters nøgler, og er derfor lavet en weak entity. Blandt vores ønskede funktioner er historik og medicinkort. Da der i en log over historik godt kan fremstå den samme medicin til den samme patient flere gange, har det været nødvendigt at tilføje en 'Historik' entitet som udover ovennævnte kan holde tidspunktet for fornyelsen af recepten.\\
Medicinkortet kan 'laves' ud fra recept entiteten givet hvilken patient der ønsker at se sit medicinkort. Det har derfor ikke været nødvendigt at lave en entitet til denne feature.\\

Vi har herefter omsat vores entiteter til tabeller på samme hvis som beskrevet i \textit{'Database Systems. The Complete Book'}\footnote{Prentice Hall, Database Systems. The Complete Book, s. 157-163}. Mange af vores entiteter kan laves direkte til en tabeller som ses nedenfor:
\begin{align*}
	&\textrm{Patient}(\textrm{\underline{CPR}, Password})\\
	&\textrm{Apotek}(\textrm{\underline{Navn}, Addresse})\\
	&\textrm{Medicin}(\textrm{\underline{Navn}, \underline{Styrke}})\\
\end{align*}
Ligeledes kan 'Diagnose' laves direkte. Denne tabel vil komme til at indeholde alle de diagnoser som kan gives, og er nødvendig da en patient kan have mere end en diagnose.\todo{High potential for change}
\begin{equation*}
\textrm{Diagnose}(\textrm{\underline{Sygdom}})
\end{equation*}
Vi har herefter vores 'Recept' entitet. Denne er weak, og henter nøgler fra de fleste andre entiteter. 'Status' attributten vil blive benyttet i overensstemmelse med vores vision om at patienten selv skal kunne følge med i receptfornyelsesprocessen. 'Udstedt' attributten vil blive benyttet af medicinkortet, og vil aldrig ændre sig da den repræsenterer hvornår patienter første gang blev ordineret med medicinen. 'Udløber' attributten vil blive benyttet til at sende reminders til patienterne.\todo{For detaljeret? Nogen grund til at forklare disse attributter?} 
\begin{equation*}
	\textrm{Recept}(\textrm{\underline{ApoNavn}, \underline{MedID}, \underline{PatCPR}, \underline{fornyet}, status, udstedt, udløber})
\end{equation*}
Hvis vi skulle lave vores relationer til om til tabeller ville der blive en en masse redundancy da alle på nær en enkelt relation vil allerede være subsets af andre tabeller. Derfor er den eneste relation som bliver lavet om til en tabel 'Behandles for'. Her vil begge nøgler være foreign keys til henholdsvis 'Patient' og 'Dignose'.
\begin{equation*}
\textrm{BehandlesFor}(\textrm{\underline{PatCPR}, \underline{Sygdom}})
\end{equation*}

\subsubsection{Known shortcommings}
Det er muligt at fornye den samme recept på forskellige apoteker, givet tuplen (\underline{ApoNavn}, \underline{MedID}, \underline{PatCPR}, \underline{fornyet}) vil være unik når apoteket ændres.
\subsection{Prototype Flask}
\todo{Add description of flask}
\subsection{Tests}
For at eliminere de værste fejl i vores protorype har vi valgt at skrive unit tests til alle vores queries defineret i \textit{models.py}. Disse kan køres med \textit{Flask test} fra kommandolinjen hvis ens 'FLASK\_APP' environment variable er sat til 'prototype.py', og man befinder sig i rod mappen. Vores tests kan også køres fra roden med kommandoen 'python -m unittest prototype/tests/test\_model.py'.\\
Vores tests forudsætter at der er lavet en testdatabase op ved navn 'prototypetests'. Testfilen opsætter selv databasen og fylder den med testdata.
\newpage
\subsection{Showcasing af prototype}
Vores prototype kan køres med \textit{Flask run} fra kommandolinjen, hvis ens 'FLASK\_APP' environment variable er sat til 'prototype.py', og man befinder sig i rod mappen. Vores prototype kan også køres fra roden med kommandoen 'python prototype.py'.\\
Vores prototype forudsætter, at der er sat en database op ved navn 'prototype'. Databasen kan opsættes ved at køre scriptsne \textit{'schema.sql'} og \textit{'schema\_ins.sql'}. Hvis disse scripts bliver benyttet, kan prototypen logges på med tre brugere med hhv. CPR-nummer 5000, 2000 og 1000, og som alle har kodeordet 'Hej'.

\begin{figure}[h!]
	\includegraphics[width=0.49\linewidth]{Materials/Prototype/Historik}
	\includegraphics[width=0.49\linewidth]{Materials/Prototype/Historik2}
	\caption{tv. ses historikken for brugeren med cprnr. 5000 og th. ses historikken for brugeren med cprnr. 1000}
\end{figure}

Vores prototype består hovedsagligt af tre funktionaliteter: \textit{Receptfornyelse, Historik} og \textit{Sundhedsdata}. Under \textit{Historik} er det muligt at se alle ens recepter og tidligere recepter, samt information omkring, hvornår recepten blev udskrevet, hvornår den udløb og hvilken medicin, der blev udskrevet.\\
Under \textit{Receptfornyelse} kan alle ens aktive recepter ses. Øverst på siden er en processlinje som beskriver, hvor langt i 'forløbet' ens medicin er kommet, altså om medicinen f.eks. først lige er blevet bestilt, om lægen har godkendt receptfornyelsen, om medicinen kan afhentes, mm. Denne processlinje understøtter vores vision om, at brugeren skal indrages i forløbet og selv kan se fremskridtene i processen. Nedenunder ses de aktive recepter samt en knap, som tillader at fornye recepten. Når brugeren klikker 'forny' ud fra en af sine recepter, så opdateres recepten til ikke længere at være aktiv, og der bliver i stedet lavet en ny recept. I prototypen er de fleste af værdierne i den nye recept hard coded, men dette ville nemt kunne gøres mere realistisk ved at lave om i vores database model.
\begin{figure}[h!]
	\includegraphics[width=0.49\linewidth]{Materials/Prototype/Receptfornyelse}
	\includegraphics[width=0.49\linewidth]{Materials/Prototype/ReceptfornyelseFornyet}
	\caption{tv. ses recepterne før fornyelse. th. ses recepterne efter 'Medicine 3' fornyelse}
\end{figure}
\newpage
Under \textit{Medicinkort} er det muligt at se alle ens aktive recepter, samt information omkring, hvilken medicin, patienter tager, dens styrke, hvad patienten er i behandling for samt et link til patienthåndbogen for denne diagnose.
\begin{figure}[h!]
	\includegraphics[width=\linewidth]{Materials/Prototype/medicinkort}
	\caption{Medicinkort for brugeren med cprnr. 5000}
\end{figure}


%\newpage

\subsection{Known shortcommings}
Der er endnu ikke implementeret business logic til at sørge for, at recepten kun kan fornys, når den er ved at udløbe, og den kan derfor fornys så ofte, det ønskes.\\
Mange værdier er hard-coded, når der indsættes en ny recept gennem receptfornyelsen. Dette ville relativt nemt kunne løses ved at lave om på database-modellen. Da dette er en prototype, hvis formål er at illustrere en mulig løsning, har dette ikke været en prioritet at løse.\\
Flere sider kan tilgås uden at være logget ind, hvilket i de fleste tilfælde bare efterlader en blank side. Dette har ikke været en prioritet at løse, da hele login delen af prototypen ville skulle udskiftes med NemID.\\

