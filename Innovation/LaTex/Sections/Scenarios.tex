Her præsenteres to scenarier, man kan forestille sig opstå ved realisering af vores vision om receptfornyelse via Min Sunhedsplatform, uniforme prøvesvar, samt samling af al information 
 på Min Sundhedsplatform, fra hhv. en brugers og en sundhedsfagligs synspunkt. \\
 Grundet manglende virksomhedsbesøg er det sundhedsfaglige synspunkt mindre konkret. 

\subsection{En brugers synspunkt}
En diabetiker er logget på Min Sundhedsplatform med formål at tjekke målinger for blodsukkerniveau. Dette kan han gøre selv uden indblanding fra lægen, da de uniforme prøvesvar er skrevet på et let tilgængeligt sprog. I samme forbindelse kommer vedkommende i tanke om, at han bør forny sin recept på insulin. Fornyelse af sin recept på insulin skal ske igennem sin personlige læge, som har eget system på egne eksterne side, som påkræver, at man tilgår lægens hjemmeside og dernæst logger ind med personlig bruger, hvorefter at en receptfornyelse kan foregå. I dette scenarie er det mere indbydende i stedet at lave sin receptfornyelse igennem Min Sundhedsplatform, idet brugeren allerede er på denne side 
\textcolor{blue}{S(skal bruge dette i forhold vores mock-ups):, og i Hovedmenuen finder menuen 'Receptfornyelse', som leder til en side, hvor receptfornyelse kan bestilles ved tryk på en knap,} 
og dermed kan gøre det hurtigere end ved at skulle tilgå et andet system, hvorfor vedkommende er mere sandsynlig at foretage sin receptfornyelse nu fremfor at udskyde dette.\\
Samtidigt husker brugeren, at vedkommende af sin diætist var blevet underlagt en kostforandring, som krævede en vis procentsats færre kulhydrater i løbet af dagen. Hertil kan brugeren nemt og hurtigt tilgå den nyttige kulhydrattæller, som findes via link til Diabetesforeningens officielle, som gør det nemmere at holde styr på sit indtag af kulhydrater. Desuden betyder, at Min Sundhedsplatform tilbyder en kulhydrattæller, at diætisten og diabetikeren har samme udgangspunkt, når de taler om patientens kulhydratindtag, fordi de ikke bruger hver sin kulhydrattæller, som evt. kunne være forskellige. 


\subsection{En Sundhedsfagligs synspunkt}
Idet receptfornyelsen igennem Min Sundhedsplatform er linket med lægens egne interne system, så oplever lægen ikke nogen forskel med hensyn til almen receptfornyelse. Dog kan man forestille sig, at noget medicin ville kræve en særlig godkendelse fra lægen således, at en receptfornyelse ikke er mulig igennem Min Sundhedsplatform, men at man i stedet kun kan "anmode" om at få en ny dosis af sin medicin. Det kunne betyde, at fremfor at lægen skulle have personlig kommunikation med patienten, omend over telefon eller en elektronisk besked, så ville lægen skulle tage stilling til mange anmodninger, som kommer igennem Min Sundhedsplatform. Dette bidrager til, at lægen i dette scenarie ville bruge færre timer på arbejdet på konsultation med patienter, men i stedet kunne bruge tid udenfor arbejdspladsen på at tage stilling til anmodninger. \\
Lægen oplever ligeså, at hun nu bruger mindre tid på personligt at besvare spørgsmål fra patienter (både over besked, telefon og ved fysisk konsultation), fordi at Min Sundhedsplatform nu tilbyder brugeren oplysninger, de behøver, som f.eks. hyppighed, hvormed medicin skal tages, hvorfor lægens tid på arbejdet nu kan bruges til "vigtige" konsultationer, fremfor hvad en patient selv ville kunne finde svar på. Dette inkluderer ligeså, at lægen ikke længere behøver at bruge konsultationstid på at forklare diverse prøvesvar, da de uniforme prøvesvar er let tilgængelige og forståelige uden lægens indblanding.\\
Fra lægesekretærens synspunkt forestiller man sig, at vores vision om receptfornyelse på Min Sundhedsplatform ville lette hendes arbejdsbyrde. I øjeblikket bruger lægesekretæren meget arbejde på at håndtere anmodninger om receptfornyelse, få dem godkendt af lægen og sende dem videre i systemet. Når receptfornyelsen er blevet implementeret, behøver hun dog ikke længere bruge arbejdstimer på at håndtere alt dette, da det gøres over Min Sundhedsplatform, således at hun nu kan bruge mere tid på at organisere besøgstider, konsultationer mm. 
