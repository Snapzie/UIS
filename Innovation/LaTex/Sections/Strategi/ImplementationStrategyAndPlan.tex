\section{Implementations-strategi og plan}
I dette afsnit ønskes at bygge et grundlag for, hvorledes vores visioner kan realiseres.
Dette indebærer forskellige faktorer, som i følgende vil undersøges:
\begin{itemize}
  \item En forståelse for projektets omfang. Her vil eksempelvis receptfornyelse igennem Min Sundhedsplatform involvere læger og deres interne systemer i hele region Sjælland og Hovedstaden, hvilket bidrager til projektets store omfang. Sammen med dette tilhører projektets finanser. Det store omfang medfører naturligvis en stor omkostning, som skal finansieres af regionen og det skal evalueres, hvorvidt vores visioner er den økonomiske investering værd. 
  \item En forståelse for diverse risici, som kunne hindre projektet. Dette inkluderer f.eks. besværlige, interne systemer hos lægerne, der kræver ekstra arbejde at koble på Min Sundhedsplatform, og hvis dette ikke virker, vil det i værste tilfælde efterlade nogen ude af stand til at benytte den nye funktionalitet til at fornye deres måske vigtige medicin, hvorfor risiko-faktoren er høj for dette projekt.  
  \item En overordnet fremgangsmåde til projektets realisering. Denne bestemmes udfra en analyse af både projektets risici, omfang og tidsbegrænsninger. Hvis tid tillader det, da vil en inkremental fremgangsmåde muligvis vise sig optimal, da en gradvis implementation af alle dele af projektet er med til at sikre en høj kvalitet af og sammenhæng mellem disse.
  \item En dybdegående analyse af diverse involverede parter og deres interesse i projektet, hvilket kan tage udgangspunkt i afsnittet om scenarier, der giver indblik i netop dette. 
\end{itemize}
