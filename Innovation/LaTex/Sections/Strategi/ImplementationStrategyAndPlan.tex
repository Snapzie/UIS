\section{Strategi og plan for implementering}
I dette afsnit ønskes at bygge et grundlag for, hvorledes vores visioner kan realiseres.
Dette indebærer forskellige faktorer, som i følgende vil undersøges:
\begin{itemize}
  \item En forståelse for projektets omfang. Her vil eksempelvis receptfornyelse igennem Min Sundhedsplatform involvere læger og deres interne systemer i hele region Sjælland og Hovedstaden, hvilket bidrager til projektets store omfang. Under samme ovevejlser tilhører projektets finanser. Det store omfang medfører naturligvis en stor omkostning, som skal finansieres af regionen og det skal evalueres, hvorvidt vores visioner er den økonomiske investering værd. 
  \item En forståelse for diverse risici og eventuelle konflikter, som kunne hindre projektet. Dette inkluderer f.eks. besværlige, interne systemer hos lægerne, der kræver ekstra arbejde at koble på Min Sundhedsplatform, og hvis dette ikke virker, vil det i værste tilfælde efterlade nogen ude af stand til at benytte den nye funktionalitet til at fornye deres måske vigtige medicin, hvorfor risiko-faktoren er høj for dette projekt. 
  En måde vores plan ville kunne håndtere tilstedeværelsen af diverse fejl og konflikter, er ved efter endt pilot-projekt at lave evaluering af denne, således at kritiske fejl opdages tidligt i processen, hvor de er relativt billige at løse kontra senere i processen.  
  %\item En overordnet fremgangsmåde til projektets realisering. Denne bestemmes ud fra en analyse af både projektets risici, omfang og tidsbegrænsninger. Hvis tid tillader det, da vil en inkremental fremgangsmåde muligvis vise sig optimal, da en gradvis implementation af alle dele af projektet er med til at sikre en høj kvalitet af og sammenhæng mellem disse.
  \item En dybdegående analyse af diverse involverede parter og deres interesse i projektet, hvilket kan tage udgangspunkt i afsnittet om scenarier, der giver indblik i netop dette. 
\end{itemize}

For vores tre prioriteringer (uniforme prøvesvar, receptfornyelse og samling af information på Min Sundhedsplatform), ønsker vi konkret følgende:
\begin{itemize}
	\item Til uniforme prøvesvar vil det primære problem ligge i den administrative process, som det ville kræve at komme til enighed omkring de uniforme prøvesvars form. Mere konkret ville samrådet blandt diverse hospitaler i regionen/landet skulle komme til enighed om dette for at opnå det ønskede resultat. Omend billigt på implementations-fronten, ville dette især være en tidskrævende, administrativ og bureakratisk proces. 
	\item Til receptfornyelse over Min Sundhedsplatform, ville der skulle laves et system, som kan interagere med diverse lægers allerede eksisterende, interne systemer. Da lægernes interne systemer ikke er garanteret at følge nogen standard, kan disse variere meget, hvorfor en implementation for at interagere med disse over Min Sundhedsplatform ville være en dyr affære at implementere. En plan til denne implementation ville være at skabe et pilot-projekt for at sikre, at de mest kritiske fejl og mangler opdages tidligst muligt i processen og derefter påbegynde en inkrementel strategi, som kan forløbe henover et års tid for at indfase regionens patienter.   

\textcolor{red}{
	Til det overstående:\\	
	Lægerne på hospitalet har ikke et intern system de bruger den del af SP som de har en adgangs rettighed til. \\
	Der er allerede integration mellem MinSP og SP.\\
	Recepten laves i SP. (Jf. Kaalund).	\\
	\\
	Der skal dog laves en forbindelse til FMK - da MinSP ikke har integration med FMK.\\
	- Se: Interview med Mette Christensen \\
	\\
	I kan også diskuteres en plan for at recept-systemet skal "vide" hvilken afdelingen på hospitalet recepten skal sendes til.\\
	\\
	- Se igen: Interview med Mette Christensen\\
	\\
	Hvis det er privat læger I snakker om har hverken Region Sjælland eller os en målsætning om at få SP og MinSP ud til privatlæger p.t.\\
}
	\item Til samling af information på Min Sundhedsplatform, skal der sikres en høj kvalitet og korrekthed af denne information, da processen ellers havde været spild og endda være en dårlig tilføjelse, hvis Min Sundhedsplatform misinformerede patienter. Derfor skal en læge eller anden relevant sundhedsfaglig person ente selv skrive informationen eller gennemgå det for faktuel korrekthed. Alternativt kan Min Sundhedsplatform linke til eksterne sider såsom steno.dk, hvilket dog ikke løser problemet med at garantere korrektheden af informationen. Den dyre del af at implementere dette, ville derfor være den tidskrævende process at fakta-checke information. 
\end{itemize}
Under udvikling af receptfornyelsesmodulet, vil hospitalerne ikke have nogen arbejdsbyrde, da implementationen af dette ligger eksternt fra hospitalerne selv. Ligeså vil arbejdet på uniforme prøvevar ikke belaste hospitalerne, da et samråd mellem hospitalerne i denne periode skal bestemme naturen af de uniforme prøvesvar, hvilket er udelukkende administrativt arbejde. I denne periode ville det derfor være oplagt at påbegynde arbejde med samling af information på Min Sundhedsplatform, da de andre processer i denne periode ikke belaster hospitalerne. Det er forventeligt, at udvikling af receptfornyelse ikke nås færdiggjort forinden, at samlingen af information på Min Sundhedsplatform realiseres. Når udvikling af receptfornyelsesmodulet så er færdiggjort, ville næste led i planen være at påbegynde pilot-projektet og derefter inkrementel-strategien for at realisere visionen om det ønskede receptfornyelsesmodul.
Når den beaukratiske proces i samrådet mellem hospitaler er færdiggjort, skal den implementeres, hvilket sker ved at lægerne og andet sundhedsfagligt personale informeres i den besluttede standard, så den fremover benyttes. Dette forventer vi sker midtvejs i processen om realisering af vores receptfornyelsesmodul.
