I innovationsfasen 
\textcolor{blue}{S: dansk: fornyelsesfasen} lå fokus på at implementere 
\textcolor{red}{S: kan man implementere en vision? } 
vores vision om 
\textcolor{blue}{S: 'vores vision om at styrke brugen af MinSP ved at tilbyde en højere grad af service og mulighed for selvbetjening for patienterne som samtidig skal være til gavn for sundhedspersonalet'}
en forbedret Min Sundhedsplatform, som vi konkluderede igennem vores forberedelsesfase.\\
Efter vores vurdering mener vi, at vores visioner 
\textcolor{blue}{S: Hvad med : 'vores forandringer' istedet for 'forkuspunkter' (visioner)}
 vil indebære en betydelig mængde arbejde fra sundhedsfagligt personale, når realiseringen af visionerne påbegyndes, men at det på sigt vil frigive arbejdstimer for personalet.\\
Vores implementationsstrategi indebærer, at vores visioner 
\textcolor{blue}{S: muligvis 'forandringer' (istedtet for 'fokuspunkter' dvs. 'visioner' )}
 igangsættes på samme tid, men grundet administrativt arbejde og udlicitering af udviklingen af receptfornyelsesmodulet,
 \textcolor{red}{S: nævner I også udlicitering og hvordan det skal forgå i afsnittet: 'Strategi og plan...'? Det høre også med der! :)} 
 vil vores realisering af visionerne sjældent have overlappende belastning på hospitalernes arbejdsbyrde.\\
For at illustrere vores visioner, er der via Python/Flask/SQL/HTML og CSS blevet lavet en prototype, som illustrerer en mulig implementation af vores visioner.\\
