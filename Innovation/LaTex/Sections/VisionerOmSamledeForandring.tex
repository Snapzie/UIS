\section{Visioner om den samlede forandring}
\subsection{Teknologi}
%
% ER Diagram over de nye funktionalitere / visioner
%
\subsubsection{It-systemer og it-platform}
Det gælder for alle visioner at it-platformen er MinSP.\\
\textbf{Vision 1: Samling af al information} \\
 %
 %
 % ? - Anders Lassen: "Læring og videnscenter. Der er allerede patienthåndbogen Jeg tror det er out-of-scope":
 %
 Denne vision ville kunne løses med en 'standardløsning', da der allerede eksistere flere troværdige informationssider, hvor det er læger der vedligeholder siderne. \\
 Viden og information om diabetes kan slås op i 'Patienthåndbogen'. Der findes allerede et link til 'Patienthåndbogen' i MinSP, men dette ligger 'gemt' i undermenuen 'Historik' under hovedmenuen 'Sundhedsdata'.\\
 Hvis det er muligt at lave en hovedmenu 'Information om dine Diagnoser / Diabetes' kunne et link til 'Patienthåndbogen' placeres her. I samme menu kan der lægges et link til 'Diabetesforeningen' der tilbyder fællesskab mellem diabetiker i form af f.eks. motivationsgrupper og diabetescafér og rådgivning til diabetiker.\\
 Endvidere kan et link til Steno Diabetes Center give video-information omkring f.eks. 'Hvordan man måler blodglukose', en diætist der fortæller om 'Kulhydrattælling' og informere om kost og motion og en overlæge der fortæller om 'Graviditetsdiabetes' m.fl. \\
 På siden er der også information til blandt andet hvis man er ny med diabetes, til gravide med diabetes, hjælp til at forstå tal, madopskrifter og meget meget mere. \\ 
 Disse 'standardløsninger' kunne være en måde at løse vision 1 på og dermed undgå en dyr løsning hvor hospitalernes sundhedspersonale skal vedligeholde informationssider mv. på MinSP.\\
 % Brugergrænseflader til vision 1 : Hovedmenu med titel 'Information om Diabetes'
 \textbf{Vision 4 (2): Uniforme Prøvesvar} \\
 % CUJannie: Jeg ønsker nemt at kunne forstå min prøvesvar. Note: En dansk forklaring af resultatet: der indeholder hvilken hospital prøven er taget på, i forhold til hvilken diagnose og hvorfor denne specifik prøve er taget / hvad den siger noget om. Med evt. en graf over udviklingen af prøvesvar f.eks. over blodsukker.\\
 %
 Denne vision ville kunnes løses ved at man til hver enkel prøvesvar knytter en forklaring af prøve-typen på dansk. Der skal angives hvor prøven er taget (hospital, læge, laboratorie).\\
 En udvidet løsning kunne være at præsentere de enkelte prøvesvare ved at tilknytte en graf der viser udviklingen af prøvesvar over tid f.eks. for blodsukker.\\
 \subsubsection{Funktioner}
 % 'Liste over de enkle it-systemers funktioner' - ?? Er dettte visionernes funktion, ??