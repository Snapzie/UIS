\section{Visioner om den samlede forandring}
Fra vores dybde analyse har vi som kandidater til forbedring af MinSP valgt følgende fokuspunkter: Fokuspunkt 8 'Receptfornyelse', fokuspunkt 1 'Samlign af al information' og fokuspunkt 4 'Uniforme prøvesvar'.
\\\\
Vi har opprioriteret fokuspunkt 8 om 'Receptfornyelse', til fokuspunkt 1, jf. argumentation nedenfor.\\
Vores oprindelige fokuspunkt 1 'Samling af al information' har vi modificeret og nedprioriteret til fokuspunkt 2. Vores argumentation herfor er, at det ville blive en for dyr løsning for sundhedspersonalet at vedligeholde og at 4 ud af 5 i spørgeskemaundersøgelsen, hvor den 5'te er neutral, ikke ønsker at MinSP skal bruges til general information omkring sygdommen diabetes, selvom dette er i modstrid med interviewsne hvor dette ønskes af de fleste.\\
Fokuspunkt 4 'Uniforme prøvesvar' har 3 prioritet, da vi i dybdeanalysen har vurderet at løsningen til fokuspunkt 2 'At skabe opmærksomhed omkring MinSP' og 3 'Overflod af fagtermer' ligger udenfor MinSP.
\\\\
Det fremgår af Region Sjællands It-strategi, jf. fokuseringsfasen, at: nye løsninger skal bruges i deres fulde omfang, patienterne og borger skal opleve en bedre servicegrad og at de i så høj grad så muligt skal kunne betjene sig selv.\\
Som det fremgår af vores brugere-undersøgelser, fra både spørgeskemaer og interviews, er der ikke så mange der kender/bruger MinSP. Spørgeskemaerne viser at 58,70\% ikke bruger MinSP og på forespørgsel oplyser 84,6\% at det er fordi ikke kender MinSP. Det fremgår også af vores interview undersøgelse hvor 4 ud af 6 ikke kender MinSP.\\
Det er et problem for Region Sjælland at patienterne ikke bruger MinSP, da det er i målsætning i Region Sjællands It-strategi at regionen ønsker fuldt udbyde af sine investeringer. \\
% ! kravet om cost/benefit skal med her
Vi valgte derfor at sætte 'At skabe opmærksomhed omkring MinSP', som anden prioritet i vores prioriteringsliste af fokuspunkter. \\
I fordybelsesfasen har vi i vores 'Mål, problemer, behov, løsninger'-liste noteret at løsningen til dette problem ligger udenfor MinSP, da vi mener at det er hospitalet selv og sundhedspersonalet der har opgaven med at systematisk at infomere om MinSP f.eks. ved at udarbejde pjæcer og fortæller om MinSP til diabetikerne via mails eller når de kommer til kontrol på hospitalet og evt. tilbyde noget læring i brugen af MinSP.\\
Samtidig mener vi også, at udvidelse af funktionalitet, forbedring af brugervenligheden og informationsniveau vil kunne understøtte en mere udbredt brug af MinSP.\\
Vi kan se ud fra vores spørgeskema, at 57.1\% forny deres recept digitalt. Dette understøttes også af vores interview undersøgelse hvor 4-5 ud af 6 diabetiker gerne vil forny deres recept digital, men bruger andre systemer som sundhed.dk eller deres læges eget system.\\
Muligheden for at forny recept, via MinSP, kan kun ske som en skrevet besked til hospitalet og funktionalitet 'Receptfornyelse' er sammentid ikke særlig synlig, da den ligger under hovedmenuen 'Meddelelser' og undermenuen 'Skriv til os' og herefter i drop-down menu 'vælg emne' findes ordet 'Receptfornyelse'. Vi mener derfor at dette også er en årsag til at patienterne ikke bruger dette modul, men andre digitale sider. Og vi mener derfor at ved at gøre tilganegen til modulet mere synligt og samtidig videre udvikle det til en funktionalitet hvor man hurtigere kan forny sine recepter samt have overblik over sine ordineret medicin, vil det motivere patienterne til at bruge dette modul i MinSP. Og fordi vi kan se at 21,43\% med sikker og 42,86\% sandsynligt ville forny deres recept via MinSP, mener vi at disse ændringer ville kunne få flere til at bruge MinSP. Derved understøttes flere af målsætningerne i Region Sjællands It-startegi punkt 2, 6 og 7 i afsnit 3.1 It-strategien s. 5-6 i 'Projektgrundlag'. Receptfornyelse vil også understøtte målsætningen om at "Brugeren skal kunne finde løsninger i samme brugervendte arbejdsgange, uden at brugerne skal opleve behov for at navigere mellem forskellige løsninger" i Region Sjællands it-infrastruktur, da MinSP som i højere grad bliver en samlet side for alle funktioner. \\ 
Overnævnte argumentation er årsag til, at vi har opprioriteret fokuspunkt 8 om 'Receptfornyelse'. 
\\\\
Vi har derfor valgt 'Receptfornyelse' som den vigtigste forbedring og valgt at udarbejde en protype for denne funktionalitet. 
\subsection{Teknologi}
%
% ! ER Diagram over de nye funktionalitere / visioner
%
\subsubsection{It-systemer og it-platform}
Det gælder for alle visioner at it-platformen er MinSP.
\\\\
\textbf{Fokuspunkt 1 (8), Receptfornyelse} \\\\
\textbf{Fokuspunkt 2 (1), Samling af al information} \\
 %
 % !  PUNKT 8 - hente information fra andre systemer - i IT-stategien skal også med her
 %
 %
 % ? - Anders Lassen: "Læring og videnscenter. Der er allerede patienthåndbogen Jeg tror det er out-of-scope":
 %
 Denne vision ville kunne løses med en 'standardløsning', da der allerede eksistere flere troværdige informationssider, hvor det er læger der vedligeholder siderne. \\
 Viden og information om diabetes kan slås op i 'Patienthåndbogen'. Der findes allerede et link til 'Patienthåndbogen' i MinSP, men dette ligger 'gemt' i undermenuen 'Historik' under hovedmenuen 'Sundhedsdata'.\\
 Hvis det er muligt at lave en menu 'Information om dine Diagnoser / Diabetes' kunne et link til 'Patienthåndbogen' placeres her. 
 I samme menu kan der lægges et link til 'Diabetesforeningen' der tilbyder fællesskab mellem diabetiker i form af f.eks. motivationsgrupper og diabetescafér og rådgivning til diabetiker.\\
 Endvidere kan et link til Steno Diabetes Center give video-information omkring f.eks. 'Hvordan man måler blodglukose', en diætist der fortæller om 'Kulhydrattælling' og informere om kost og motion og en overlæge der fortæller om 'Graviditetsdiabetes' m.fl. \\
 På siden er der også information til blandt andet hvis man er ny med diabetes, til gravide med diabetes, hjælp til at forstå tal, madopskrifter og meget meget mere. \\ 
 Disse 'standardløsninger' kunne være en måde at løse vision 1 på og dermed undgå en dyr løsning hvor hospitalernes sundhedspersonale skal vedligeholde informationssider mv. på MinSP.
 \\\\
 \textbf{Fokuspunkt 3 (4), Uniforme Prøvesvar} \\
 Denne vision ville kunnes løses ved at man til hver enkel prøvesvar knytter en forklaring af prøve-typen på dansk. Der skal angives hvor prøven er taget (hospital, læge, laboratorie).\\
  % CUJannie: Jeg ønsker nemt at kunne forstå min prøvesvar. Note: En dansk forklaring af resultatet: der indeholder hvilken hospital prøven er taget på, i forhold til hvilken diagnose og hvorfor denne specifik prøve er taget / hvad den siger noget om. Med evt. en graf over udviklingen af prøvesvar f.eks. over blodsukker.\\
 En udvidet løsning kunne være at præsentere de enkelte prøvesvare ved at tilknytte en graf der viser udviklingen af prøvesvar over tid f.eks. for blodsukker.\\
 \subsubsection{Funktion}  % ? - Indenholde: 'Liste over de enkle it-systemers funktioner'
 \textbf{Fokuspunkt 1 (8), Receptfornyelse}\\
 Receptfornyelse er en funktionalitet der skal ny-udvikles. Kravet til funktionaliteten skal være så patienter, i MinSP, selv kan aktivere fornyelse af en eller flere recepter for deres ordinerede medicin. \\
 Receptfornyelsen skal være synlig og skal derfor placeres som en hovedmenu øverst på MinSP. Undermenuen til hovedmenuen 'Receptfornyelse' skal indeholde: 'Forny recept', 'Medicinkort' og 'Historik'.\\
 Man skal kunne følge gangen i receptfornyelsen fra status 'Medicin bestilt' til 'Medicin kan hentes på Apoteket'. Man skal også kunne vælge modtager-apotek, med to valgmuligheder, og om man ønsker en påmindelse om receptfornyelse og i hvilken form. \\ 
 Recept skal kun kunne fornys, når sidst udleveret dosis er ved slippe op.  \\
 'Medicinkortet' skal indeholde en liste over ordineret medicin og 'Historik' skal indeholde en liste over alle udleveringer af medicin til dato.\\
 Vi vurdere at 'Receptfornyelse' er kompliceret funktionalitet, da den skal ny-udvikles og der er ikke noget eksisterende funktionalitet der kan genbruges. Funktionalitet er også kompliceret fordi sikkerhed skal være høj i forhold at der ikke må udskrives for meget medicin og de kun er den ordinerede medicin der må fornys. \\
 Udvikling af status linje for gangen i receptfornyelsen vil også være kompliceret, fordi registreringen skal overføres af sundhedspersonalet fra Sundhedsplatformen til MinSP.\\
 Modtagerapotek skal kunne vælges i forhold til bopæl og det vil også være kompliceret. \\
 Påmindelses vil kræve at der beregnes en dato for fornyelse af recept i forhold til tid eller dosis ved sidste fornyelse, dette er igen kompliceret at udvikle.\\
 Der skal være en database hvor oplysninger om ordineret medicin, historikken for udleveret medicin og receptfornyelser gemmes. \\ 
 Til patient database skal der tilknyttes attributterne 'Primær apotek' og 'Sekundær apotek'.\\
 \subsubsection{Brugergrænseflader}
 Brugergrænsefladerne skal designes så de overholder GUI-guidelines for god interaktions design.
 \\\\  %  Benyon s. 88 
 \textbf{Fokuspunkt 1 (8), Receptfornyelse}

  % Brugergrænseflader til Fokuspunkt 2 : Hovedmenu med titel 'Information om Diabetes'