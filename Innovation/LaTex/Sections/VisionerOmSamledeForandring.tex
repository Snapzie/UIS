\section{Visioner om den samlede forandring}
Fra vores dybde analyse har vi som kandidater til forbedring af MinSP valgt følgende fokuspunkter: Fokuspunkt 8 'Receptfornyelse', fokuspunkt 1 'Samlign af al information' og fokuspunkt 4 'Uniforme prøvesvar'.
\\\\
Vi har opprioriteret fokuspunkt 8 om 'Receptfornyelse', til fokuspunkt 1, jf. argumentation nedenfor.\\
Vores oprindelige fokuspunkt 1 'Samling af al information' har vi modificeret og nedprioriteret til fokuspunkt 2. Vores argumentation herfor er, at det ville blive en for dyr løsning for sundhedspersonalet \todo[]{Hvad tænker du på vil blive dyrt?} at vedligeholde og at 4 ud af 5 i spørgeskemaundersøgelsen, hvor den 5'te er neutral, ikke ønsker at MinSP skal bruges til general information omkring sygdommen diabetes, selvom dette er i modstrid med interviewsne hvor dette ønskes af de fleste.\\
Fokuspunkt 4 'Uniforme prøvesvar' har 3 prioritet, da vi i dybdeanalysen har vurderet at løsningen til fokuspunkt 2 'At skabe opmærksomhed omkring MinSP' og 3 'Overflod af fagtermer' ligger udenfor MinSP.\todo{Jeg forstår ikke sammenhængen?}
\\\\
Det fremgår af Region Sjællands It-strategi, jf. fokuseringsfasen, at: nye løsninger skal bruges i deres fulde omfang, patienterne og borger skal opleve en bedre servicegrad og at de i så høj grad så muligt skal kunne betjene sig selv.\\
Som det fremgår af vores brugere-undersøgelser, fra både spørgeskemaer og interviews, er der ikke så mange der kender/bruger MinSP. Spørgeskemaerne viser at 58,70\% ikke bruger MinSP og på forespørgsel oplyser 84,6\% at det er fordi ikke kender MinSP. Det fremgår også af vores interview undersøgelse hvor 4 ud af 6 ikke kender MinSP.\\
Det er et problem for Region Sjælland at patienterne ikke bruger MinSP, da det er i målsætning i Region Sjællands It-strategi at regionen ønsker fuldt udbyde af sine investeringer. \\
Vi valgte derfor at sætte 'At skabe opmærksomhed omkring MinSP', som anden prioritet i vores prioriteringsliste af fokuspunkter. \\
I fordybelsesfasen har vi i vores 'Mål, problemer, behov, løsninger'-liste noteret at løsningen til dette problem ligger udenfor MinSP, da vi mener at det er hospitalet selv og sundhedspersonalet der har opgaven med at systematisk at infomere om MinSP f.eks. ved at udarbejde pjæcer og fortæller om MinSP til diabetikerne via mails eller når de kommer til kontrol på hospitalet og evt. tilbyde noget læring i brugen af MinSP.\\
Samtidig mener vi også, at udvidelse af funktionalitet, forbedring af brugervenligheden og informationsniveau vil kunne understøtte en mere udbredt brug af MinSP.\\
Vi kan se ud fra vores spørgeskema, at 57.1\% forny deres recept digitalt. Dette understøttes også af vores interview undersøgelse hvor 4-5 ud af 6 diabetiker gerne vil forny deres recept digital, men bruger andre systemer som sundhed.dk eller deres læges eget system.\\
Muligheden for at forny recept, via MinSP, kan kun ske som en skrevet besked til hospitalet og funktionalitet 'Receptfornyelse' er sammentid ikke særlig synlig, da den ligger under hovedmenuen 'Meddelelser' og undermenuen 'Skriv til os' og herefter i drop-down menu 'vælg emne' findes ordet 'Receptfornyelse'. Vi mener derfor at dette også er en årsag til at patienterne ikke bruger dette modul, men andre digitale sider. Og vi mener derfor at ved at gøre tilganegen til modulet mere synligt og samtidig videre udvikle det til en funktionalitet hvor man hurtigere kan forny sine recepter samt have overblik over sine ordineret medicin, vil det motivere patienterne til at bruge dette modul i MinSP. Og fordi vi kan se at 21,43\% med sikker og 42,86\% \todo{Referencer}sandsynligt ville forny deres recept via MinSP, mener vi at disse ændringer ville kunne få flere til at bruge MinSP. Derved understøttes flere af målsætningerne i Region Sjællands It-startegi punkt 2, 6 og 7 i afsnit 3.1 It-strategien s. 5-6 i 'Projektgrundlag'. Receptfornyelse understøtter også målsætningen i Region Sjællands it-infrastruktur om at "Brugeren skal kunne finde løsninger i samme brugervendte arbejdsgange, uden at brugerne skal opleve behov for at navigere mellem forskellige løsninger", da MinSP i højere grad bliver en samlet side for alle funktioner. \\ 
Overnævnte argumentation er årsag til, at vi har opprioriteret fokuspunkt 8 om 'Receptfornyelse'. 
\\\\
Vi har derfor valgt 'Receptfornyelse' som den vigtigste forbedring og valgt at udarbejde en protype for denne funktionalitet. 
\subsection{Teknologi}
%
% ! ER Diagram over de nye funktionalitere / visioner
%
\subsubsection{It-systemer og it-platform}
Det gælder for alle visioner at it-platformen er Sundhedsplatformen.\\\\
\textbf{Fokuspunkt 1, Receptfornyelse} \\
It-systemet er udover MinSP også Sundhedsplatformen og muligvis apotekernes systemer, da der skal være integration i mellem disse systemer.
\\\\
\textbf{Fokuspunkt 2, Samling af al information} \\
 %
 % ? - Anders Lassen: "Læring og videnscenter. Der er allerede patienthåndbogen Jeg tror det er out-of-scope":
 %
 Denne funktionalitet ville kunne løses med en 'standardløsning', da der allerede eksisterer flere troværdige informationssider, hvor det er læger der vedligeholder siderne. \\
 Viden og information om diabetes kan slås op i 'Patienthåndbogen'. Der findes allerede et link til 'Patienthåndbogen' i MinSP, men dette ligger 'gemt' i undermenuen 'Historik' under hovedmenuen 'Sundhedsdata'.\\
 Hvis det er muligt at lave en menu 'Information om dine Diagnoser / Diabetes' kunne et link til 'Patienthåndbogen' placeres her. 
 I samme menu kan der lægges et link til 'Diabetesforeningen' der tilbyder fællesskab mellem diabetikere i form af f.eks. motivationsgrupper og diabetescafér og rådgivning til diabetikere.\\
 Endvidere kan et link til 'Steno Diabetes Center' give video-information omkring f.eks. 'Hvordan man måler blodglukose', en diætist der fortæller om 'Kulhydrattælling' og informerer om kost og motion og en overlæge der fortæller om 'Graviditetsdiabetes' m.fl. \\
 På siden er der også information om blandt andet det at være ny med diabetes, til gravide med diabetes, hjælp til at forstå tal, madopskrifter og meget meget mere. \\ 
 Disse 'standardløsninger' kunne være en måde at løse funktionalitet 'Samling af al information' på og dermed undgå en dyr løsning hvor hospitalernes sundhedspersonale skal vedligeholde informationssider mv. på MinSP.
 \\\\
 \textbf{Fokuspunkt 3, Uniforme Prøvesvar} \\
 It systemet er udover MinSP også Sundhedsplatformen.
 \subsubsection{Funktion}  % ? - Indenholde: 'Liste over de enkle it-systemers funktioner'
 \textbf{Fokuspunkt 1, Receptfornyelse}\\
 Receptfornyelse er en funktionalitet (vision - ?) der skal ny-udvikles. Kravet til funktionaliteten skal være så patienter, i MinSP, selv kan aktivere fornyelse af en eller flere recepter for deres ordinerede medicin. \\
 Receptfornyelsen skal være synlig og skal derfor placeres som en hovedmenu øverst på MinSP. Undermenuen til hovedmenuen 'Receptfornyelse' skal indeholde: 'Forny recept', 'Medicinkort' og 'Historik'.\\
 Man skal kunne følge gangen i receptfornyelsen fra status 'Medicin bestilt' til 'Medicin kan hentes på Apoteket'. Man skal også kunne vælge modtager-apotek, med to valgmuligheder, og om man ønsker en påmindelse om receptfornyelse og i hvilken form. \\ 
 Recept skal kun kunne fornys, når sidst udleveret dosis er ved slippe op.  \\
 'Medicinkortet' skal indeholde en liste over ordineret medicin og 'Historik' skal indeholde en liste over alle udleveringer af medicin til dato.\\
 Vi vurdere at 'Receptfornyelse' er kompliceret funktionalitet, da den skal ny-udvikles og der er ikke noget eksisterende funktionalitet der kan genbruges. Funktionalitet er også kompliceret fordi sikkerhed skal være høj i forhold at der ikke må udskrives for meget medicin og de kun er den ordinerede medicin der må fornys. \\
 Udvikling af status linje for gangen i receptfornyelsen vil også være kompliceret, fordi registreringen skal overføres af sundhedspersonalet fra Sundhedsplatformen til MinSP.\\
 Modtagerapotek skal kunne vælges i forhold til bopæl og det vil også være kompliceret. \\
 Påmindelses vil kræve at der beregnes en dato for fornyelse af recept i forhold til tid eller dosis ved sidste fornyelse, dette er igen kompliceret at udvikle.\\
 Der skal være en database hvor oplysninger om ordineret medicin, historikken for udleveret medicin og receptfornyelser gemmes. \\ 
 Til patient database skal der tilknyttes attributterne 'Primær apotek' og 'Sekundær apotek'.
 \\\\
 \textbf{Fokuspunkt 2, Samling af al information} \\
 Et forslag til at imødekomme vision om en 'Samling af al information' kunne være at tilføje under hovedmenuen Profil at tilføje en undermenu 'Information om dine diagnoser' og så under denne undermenu ligge link til f.eks. Patienthåndbogen, Diabetesforeningen og Steno Diabetes Center.
 \\\\ 
 \textbf{Fokuspunkt 3, Uniforme Prøvesvar} \\
 Denne funktionalitet ville kunne løses ved at man til hver enkel prøvesvar knytter en forklaring af prøve-typen på dansk. Der skal udover angives hvor prøven er taget (hospital, læge, laboratorie).\\
 Prøvesvarende skal holdes adskilt pr. diagnoses hvis patienten har flere diagnoser.\\
 En udvidet løsning kunne være at præsentere de enkelte prøvesvare ved at tilknytte en graf der viser udviklingen af prøvesvar over tid f.eks. for blodsukker.\\
 Eksisterende database / tabel med prøvesvar skal have tilknyttet en attribut med forklarende tekst om hvad prøven viser på dansk, og en attribut med navn på hospital / laboratorie. Derfor skal være en tabel med hospitals / laboratorie navne som navnet kan hentes fra.\\ 
 Tekstbeskrivelsen kan være standart tekst pr. prøvetype, men hospitalet / laboratoriet hvor prøven er taget kan variere. Dette skal derfor indrapporteres af personalet når prøvesvar indrapporteres. \\
 At holde prøvesvar adskilt pr. diagnose kræver at der i prøvesvar tabellen skal være en attribut for diagnose som så for hvert prøvesvar skal udfyldes med relevant diagnose af sundhedspersonalet.\\
 Der skal kunne udtrækkes historik fordi de prøvetyper hvor de er relevant at kunne vise en graf og kode der generere en graf ud fra talene.
 \subsubsection{Brugergrænseflader} % Krav til Brugergrænseflader
 Brugergrænsefladerne skal designes så de overholder GUI-guidelines for god interaktions design. %  Benyon s. 88 
 \\\\
 \textbf{Fokuspunkt 1, Receptfornyelse} \\
 Implementering af receptfornyelses funktionaliteten vil kræve integration med Sundhedsplatformens medicinmodul og apotekerenes systemer.\\ 
 Forslag til brugergrænseflade for patienten vil fremgå af nedenstående Mock-up's:\\
 % Mock-up 
 % Scenerier
 \textbf{Fokuspunkt 2, Samling af al information} \\
 Forslag til brugergrænseflade for patienten vil fremgå af nedenstående Mock-up's:\\
 % Mock-up 
 \textbf{Fokuspunkt 3, Uniforme Prøvesvar} \\
 Forslag til brugergrænseflade for patienten vil fremgå af nedenstående Mock-up's:\\
 % Mock-up 
  \subsection{Arbejdets organisering}
\textbf{Fokuspunkt 1, Receptfornyelse} \\
En automatisk receptfornyelse vil muligvis ændre arbejdsgangen for sundhedspersonalet. \\ 
Vi tænker at lægen forsat stadigvæk skal tjekke at ordineringen er korrekt i forhold til journalen, men at sekretæren slipper for at skrive besked tilbage til patienten, nu kan patienten selv se forløbet i status.
\\\\
\textbf{Fokuspunkt 2, Samling af al information} \\
Hvis funktionaliteten implementeres via link, vil der ikke være nogen ændringer til arbejdsgange og arbejdets organisering. \\
Informationslinkene kan også være tilgavn for sundhedspersonalet. \\
Hvis ikke funktionaliteten implementeres via link, men som en ny informationside, vil den skulle vedligeholdes af sundhedspersonalet, og det ville være en ekstra opgave som sundhedspersonalet (læger, diætister og sygeplejeskær m.fl.) herved bliver pålagt.
\\\\
\textbf{Fokuspunkt 3, Uniforme Prøvesvar} \\
Funktionaliteten vil give ekstra arbejde for lægesekretær eller laboratorier medarbejder i forhold til indrapporteringen af prøvesvar, da der vil flere informationer der skal indtastes. Der vil forenden være ekstra arbejde for lægen da det er denne der skal informere laboratorie-personale / lægesekretær om hvilken diagnose de enkle prøvesvar vedrøre.\\
Tilgengæld vil prøvesvarende blive mere forståelig for patienterne og det lette arbejdet for sundhedspersonalet med at besvare spørgsmål og forklare prøvesvar for patienterne.
\subsection{Kvalifikationsbehov}
\textbf{Fokuspunkt 1, Receptfornyelse} \\
  Receptfornyelse funktionaliteten vil ikke kræve nogen større oplæringen af sundhedspersonalet, kun en introduktion til hvordan den nye funktionalitet fungere.
  \\\\
  \textbf{Fokuspunkt 2, Samling af al information} \\
  Hvis funktionaliteten implementeres via link, vil det ikke kræve nogen oplæring af sundhedspersonalet.
  \\\\
  \textbf{Fokuspunkt 3, Uniforme Prøvesvar} \\
  Vores vurdering er at ikke at der kraves ekstra kvalifikationer.