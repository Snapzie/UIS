%MoSCoW Prioritering
\subsection{MoSCoW Prioritering}
Nedenstående MoSCoW prioritering er udarbejdet på baggrund af future user stories.\\
Den er blevet brugt til at vurdere vores prioriterings-liste fra fordybelsesfasen og som argument for omprioritering.\\
Vi har ikke nogen "Must-have" prioriteringer, da "Must-have" dækker over minimumsmængden af funktioner, som systemet skal understøtte for, at det er brugbart.\\
MinSP er i dag brugbar til de grundlægende funktioner, som at kunne logge ind, se sine prøvesvar, sende beskeder. læse beskeder og se aftaler. \\
Derfor har vi ingen prioriteringer.\\
"Should-have" er ikke så tidskritiske som "Must-have", da der er en anden måde at løse funktionaliteten på, men de kan være lige så essentielle.\\
Derfor danner vores "Should-have"-prioriteringer grundlag for valg af protype.\\
"Could-have" er prioriteter, der kun tages med, hvis der er tid og ressourcer til det, da de kan forbedre brugeroplevelsen, men ikke er streng nødvendige. \\
"Won't-have"-prioriteterne tages ikke med i det videre arbejde denne gang.
\subsubsection*{Must have}
\subsubsection*{Should have}
\begin{itemize}
\item CU1: Ønsker jeg mulighed for at forny min recept gennem MinSP, så jeg kun skal ind ét sted og ikke skal vente i telefonen.
\item CUKaren: Ønsker jeg, at jeg direkte ledes det rigtig sted hen i menuen og ikke af omveje, så jeg ikke leder forgæves. \\
Note: Brugervenlighed - i forhold til receptfornyelse samt booking af aftaler.
\item CUKaren: Ønsker jeg at kunne læse information omkring diabetes på MinSP, så jeg ikke skal søge flere steder. \\
Note: information kan være om diabetes og om kost.
\item CUJannie: Ønsker jeg at kunne forstå min prøvesvar, så jeg ved hvad, de betyder. \\
Note: En dansk forklaring af resultatet, hvilket hospital prøven er taget på og til hvilken diagnose.
\end{itemize}
\subsubsection*{Could have}
\begin{itemize}
\item CUErnest: Ønsker jeg at modtage reminders om kost og motion for at blive motiveret.
\item CUErnest: Ønsker jeg at kunne se hvor indhold i min journal stammer fra, så jeg bedre kan overskue det.
\item CUPeter: Ønsker jeg påmindelse om receptfornyelse, konsultation mv., så jeg ikke skal huske på det selv.
\item CUJulia: Ønsker jeg, at oplysninger om mine diagnoser er adskilte for at opnå en mere personlig MinSP.\\
% Skal være noget der giver kunde tilfredshed, med Små omkostninger - de er ikke nødvendige, men kan forbedre uder erfaring og brugertilfredshed
Note: kunne vælge mellem e-boks, email, sms.
\end{itemize}
\subsubsection*{Won't have}
\begin{itemize}
\item CUKaren: Jeg ønsker at kunne tilgå et diskussionsforum for patienter for at møde ligesindede.
\end{itemize}
