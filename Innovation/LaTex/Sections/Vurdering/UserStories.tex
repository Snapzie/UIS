%User stories:
%
%Independece
%Der må ikke være afhægihed mellem histrierne
%Afhægihed mellem historierne: giver planlægningens / prioterings problemer
%
%Negotiable
%Kort beskrivelse af funktionalitet
%Detajerne omkring hvordan det skal skal ikke være med - det er til forhandling.
%De er husker om hvad vi skal have med.
%Tilføj kort 'Note', hvis der detajer/overvejelser der er vigtige at have med.
%
%Vauluable to Purchasers or Users
%Skal skrives så man ser fordel for brugernes synspunkt - ikke kundens /programmørens
%Brugernes antagelser om interfacet skal ikke med
%Teknologi antagelse skal ikke med
%
%Estimatable
%De skal kunne estimeres dvs. vurdere hvad det vil koste at udvikle for at udvikle det.
%User stories skal bruges i konsekvens analysen når man skal estimere omkostninger.
%
%Small
%Det skal være konkret. Feks det må ikke være ' en bruger skal kunne bestille en rejse ' (epic) det skal være splittet op:
%          ' en bruger skal kunne kontakte rejse bureauet '
%
%Test
%Man skal kunne skrive test ud fra dem
\subsection{User Stories}
\subsubsection{\textbf{Current User Stories}} % Current User stories er for hver af de relevante aktøre i Domænet
\subsubsection*{Som patient:}
- kan jeg se min Journal. Note: Journal har journalnotater. \\
- kan jeg se min diagnose-oversigt.\\
- kan jeg se min målings-oversigt.\\
- kan jeg modtage beskeder og se dem i min Indbakke.\\
- kan jeg sende en besked og se dem i min Udbakke.\\
- kan jeg modtager påmindelser.\\
- kan jeg se mine aftaler. \\
- kan jeg mine prøvesvar.\\
- kan jeg udfylde et spørgeskema.\\
- kan jeg give en fuldmagt. \\
- har jeg kontakt med sundhedsfaglige personer. Note: Sundhedsfaglig person kan være en læge. \\
- som patient kan jeg se hvem der har været logget på min journal.
\subsubsection*{Som sundhedsfaglig person:}
- har jeg kontaktansvar for flere patienter.\\
- skal jeg kunne logge på og skrive notater i patients journal.\\
\subsubsection{\textbf{Furture User Stories}}
\subsubsection*{Som diabetes patient:}
- CUKaren: ønsker jeg at kunne læse information omkring diabetes på MinSP. Note: information kan være om diabetes og om kost.\\
- CUJulia: ønsker jeg at oplysninger om mine diagnoser er adskilte. Note: for at opnå en mere personlig MinSP.\\
- CUKaren: ønsker jeg at jeg direkte ledes det rigtig sted hen i menuen og ikke af omveje. Note: Brugervenlighed f.eks. i forhold til receptfornyelse.\\ 
- CUErnest: ønsker jeg at modtage reminders om kost og motion. Note: Motivation. \\
- CUJannie: ønsker jeg nemt at kunne forstå min prøvesvar. Note: En dansk forklaring af resultatet: der indeholder hvilken hospital prøven er taget på, i forhold til hvilken diagnose og hvorfor denne specifik prøve er taget / hvad den siger noget om. Med evt. en graf over udviklingen af prøvesvar f.eks. over blodsukker.\\
- CUErnest: ønsker jeg at kunne se hvor indhold i min journal stammer fra.
- CUPeter: ønsker jeg påmindelse om receptfornyelse. Note: kunne vælge mellem e-boks, email, sms.
- CUKaren: ønsker jeg at kunne tilgå et diskussionsforum for patienter.\\
%
% Ydligere funktioner nogen i gruppen har snakket om /forslået:
% - Skype / videosamtale med sundhespersonale / læge
% - Nyt udstyr der måler blodsukker og sender det til MinSP => giver en graf over det (der kan f.eks. laves en Mock-up),
% -