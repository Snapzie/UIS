%User stories:
%
%Independece
%Der må ikke være afhægihed mellem histrierne
%Afhægihed mellem historierne: giver planlægningens / prioterings problemer
%
%Negotiable
%Kort beskrivelse af funktionalitet
%Detajerne omkring hvordan det skal skal ikke være med - det er til forhandling.
%De er husker om hvad vi skal have med.
%Tilføj kort 'Note', hvis der detajer/overvejelser der er vigtige at have med.
%
%Vauluable to Purchasers or Users
%Skal skrives så man ser fordel for brugernes synspunkt - ikke kundens /programmørens
%Brugernes antagelser om interfacet skal ikke med
%Teknologi antagelse skal ikke med
%
%Estimatable
%De skal kunne estimeres dvs. vurdere hvad det vil koste at udvikle for at udvikle det.
%User stories skal bruges i konsekvens analysen når man skal estimere omkostninger.
%
%Small
%Det skal være konkret. Feks det må ikke være ' en bruger skal kunne bestille en rejse ' (epic) det skal være splittet op:
%          ' en bruger skal kunne kontakte rejse bureauet '
%
%Test
%Man skal kunne skrive test ud fra dem
\subsection{User Stories}
\subsubsection{Current User Stories} % Current User stories er for hver af de relevante aktøre i Domænet
Current User Stories er udarbejdet ud fra ER diagrammet over domainet for MinSP. \footnote{Bilag 4} 
\subsubsection*{Som patient:}
\begin{itemize}
\item kan jeg se min Journal. Note: Journal har journalnotater. 
\item kan jeg se min diagnose-oversigt.
\item kan jeg se min målings-oversigt.
\item kan jeg modtage beskeder og se dem i min Indbakke.
\item kan jeg sende en besked og se dem i min Udbakke.
\item kan jeg modtager påmindelser.
\item kan jeg se mine aftaler. 
\item kan jeg mine prøvesvar.
\item kan jeg udfylde et spørgeskema.
\item kan jeg give en fuldmagt. 
\item har jeg kontakt med sundhedsfaglige personer. 
\item som patient kan jeg se hvem der har været logget på min journal.
\end{itemize}
\subsubsection*{Som sundhedsfaglig person:}
\begin{itemize}
\item har jeg kontaktansvar for flere patienter.
\item skal jeg kunne logge på og skrive notater i patients journal.
\end{itemize}
\subsubsection{Furture User Stories}
Furture user stories er udarbejdet ud fra vores bruger-undersøgelser.
\subsubsection*{Som diabetes patient:}
\textbf{Fra interview}\\
\begin{itemize}
\item CUKaren: Ønsker jeg at kunne læse information omkring diabetes på MinSP, så jeg ikke skal søge flere steder. \\
Note: information kan være om diabetes og om kost.
\item CUJulia: Ønsker jeg at oplysninger om mine diagnoser er adskilte, for at opnå en mere personlig MinSP. 
\item CUKaren: Ønsker jeg at jeg direkte ledes det rigtig sted hen i menuen og ikke af omveje, så jeg ikke leder forgæves.\\ 
Note: Brugervenlighed f.eks. i forhold til receptfornyelse.
\item CUErnest: Ønsker jeg at modtage reminders om kost og motion, for at blive motiveret.
\item CUJannie: Ønsker jeg at kunne forstå min prøvesvar, så jeg ved hvad de betyder. \\
Note: En dansk forklaring af resultatet, hvilken hospital prøven er taget på og til hvilken diagnose. 
\item CUErnest: Ønsker jeg at kunne se hvor indhold i min journal stammer fra, så jeg bedre kan overskue det.
\item CUPeter: Ønsker jeg påmindelse om receptfornyelse, konsultation mv., så jeg ikke skal huske på det selv.\\ 
Note: kunne vælge mellem e-boks, email, sms.
\item CUKaren: Ønsker jeg at kunne tilgå et diskussionsforum for patienter, for at møde ligesindede.
\end{itemize}
\textbf{Fra spørgeskema-undersøgelse}\\
\begin{itemize}
\item CU1: Ønsker jeg mulighed for at forny min recept gennem MinSP, så jeg kun skal ind et sted og ikke skal vente i telefonen.
\end{itemize}
%
% Ydligere funktioner nogen i gruppen har snakket om /forslået:
% - Skype / videosamtale med sundhespersonale / læge
% - Nyt udstyr der måler blodsukker og sender det til MinSP => giver en graf over det (der kan f.eks. laves en Mock-up),
% -