%User stories:
%
%Independece
%Der må ikke være afhægihed mellem histrierne
%Afhægihed mellem historierne: giver planlægningens / prioterings problemer
%
%Negotiable
%Kort beskrivelse af funktionalitet
%Detajerne omkring hvordan det skal skal ikke være med - det er til forhandling.
%De er husker om hvad vi skal have med.
%Tilføj kort 'Note', hvis der detajer/overvejelser der er vigtige at have med.
%
%Vauluable to Purchasers or Users
%Skal skrives så man ser fordel for brugernes synspunkt - ikke kundens /programmørens
%Brugernes antagelser om interfacet skal ikke med
%Teknologi antagelse skal ikke med
%
%Estimatable
%De skal kunne estimeres dvs. vurdere hvad det vil koste at udvikle for at udvikle det.
%User stories skal bruges i konsekvens analysen når man skal estimere omkostninger.
%
%Small
%Det skal være konkret. Feks det må ikke være ' en bruger skal kunne bestille en rejse ' (epic) det skal være splittet op:
%          ' en bruger skal kunne kontakte rejse bureauet '
%
%Test
%Man skal kunne skrive test ud fra dem
\subsection{User Stories}
\textcolor{red}{Husker: Der mangler tekst omkring at disse er lavet fra 1. ER Diagram}\\
\subsubsection{Current User Stories} % Current User stories er for hver af de relevante aktøre i Domænet
\subsubsection*{Som patient:}
- kan jeg se min Journal. Note: Journal har journalnotater. \\
\\
- kan jeg se min diagnose-oversigt.\\
\\
- kan jeg se min målings-oversigt.\\
\\
- kan jeg modtage beskeder og se dem i min Indbakke.\\
\\
- kan jeg sende en besked og se dem i min Udbakke.\\
\\
- kan jeg modtager påmindelser.\\
\\
- kan jeg se mine aftaler. \\
\\
- kan jeg mine prøvesvar.\\
\\
- kan jeg udfylde et spørgeskema.\\
\\
- kan jeg give en fuldmagt. \\
\\
- har jeg kontakt med sundhedsfaglige personer. \\
\\
- som patient kan jeg se hvem der har været logget på min journal.
\subsubsection*{Som sundhedsfaglig person:}
- har jeg kontaktansvar for flere patienter.\\
\\
- skal jeg kunne logge på og skrive notater i patients journal.\\
\\
\textcolor{red}{Husk: Tekst til furture user stories ud fra interviews og spørgeskema}
\subsubsection{Furture User Stories}
\subsubsection*{Som diabetes patient:}
\textbf{Fra interview}\\\\
- CUKaren: Ønsker jeg at kunne læse information omkring diabetes på MinSP, så jeg ikke skal søge flere steder. \\
Note: information kan være om diabetes og om kost.\\
\\
- CUJulia: Ønsker jeg at oplysninger om mine diagnoser er adskilte, for at opnå en mere personlig MinSP. \\
\\
- CUKaren: Ønsker jeg at jeg direkte ledes det rigtig sted hen i menuen og ikke af omveje, så jeg ikke leder forgæves.\\ 
Note: Brugervenlighed f.eks. i forhold til receptfornyelse.\\ 
\\
- CUErnest: Ønsker jeg at modtage reminders om kost og motion, for at blive motiveret.\\
\\
- CUJannie: Ønsker jeg at kunne forstå min prøvesvar, så jeg ved hvad de betyder. \\
Note: En dansk forklaring af resultatet, hvilken hospital prøven er taget på og til hvilken diagnose. \\
\\
- CUErnest: Ønsker jeg at kunne se hvor indhold i min journal stammer fra, så jeg bedre kan overskue det.\\
\\
- CUPeter: Ønsker jeg påmindelse om receptfornyelse, konsultation mv., så jeg ikke skal huske på det selv.\\ 
Note: kunne vælge mellem e-boks, email, sms.\\
\\
- CUKaren: Ønsker jeg at kunne tilgå et diskussionsforum for patienter, for at møde ligesindede.\\
%
% Ydligere funktioner nogen i gruppen har snakket om /forslået:
% - Skype / videosamtale med sundhespersonale / læge
% - Nyt udstyr der måler blodsukker og sender det til MinSP => giver en graf over det (der kan f.eks. laves en Mock-up),
% -