\section{Vurdering af prioriteter}
Efter grundig vurdering af vores prioritering af fokuspunkter fra dybdeanalysen, har vi valgt at ændre i denne prioritering og har i stedet, som de tre vigtigste fokuspunkter i vores videre arbejde med forbedring af MinSP. valgt følgende:
1. 'Receptfornyelse'\\ 
2. 'Samling af al information'\\ 
3. 'Uniforme prøvesvar'.\\
\\\\
Vores oprindelige prioritet 1 'Samling af al information' har vi ændret til prioritet 2.
Vores argumentation herfor er, at det ville blive en for dyr løsning for sundhedspersonalet at vedligeholde, hvis vi som vi havde beskrevet, lavede funktionaliteten som en ny-udvikling.  
Ligeledes viste vores spørgeskemaundersøgelse, at 4 ud af 5 diabetikere, hvor den 5'te var neutral, ikke havde ønske om at MinSP skulle bruges til generel information omkring sygdommen diabetes. Dette er dog i modstrid med vores interview-undersøgelse, hvor dette ønskes af de fleste. Derfor har vi beholdt den som prioritet 2, men modificeret til en løsning, der anvender en 'standardløsning' med links. \\
Vores oprindelige prioritet 2 'At skabe opmærksomhed omkring MinSP' og prioritet 3 'Overflod af fagtermer' har vi vurderet til at ligge udenfor MinSP. Vores oprindelige prioritet 4 'Uniforme prøvesvar' bliver derfor rykket op til prioritet 3.
\\ \\
Det fremgår af Region Sjællands It-strategi \footnote{Projektgrundlag, s. 5, afsnit 3.1 IT Strategien}, at "...nye løsninger skal bruges i deres fulde omfang", og at patienterne og borgere skal opleve en bedre servicegrad, og at de, i så høj grad som muligt, skal kunne betjene sig selv.\\
Men af vores brugere-undersøgelser fremgår det, fra både spørgeskemaer og interviews, at der ikke er så mange der kender/bruger MinSP. Spørgeskemaerne viser, at 58,70\% ikke bruger MinSP, og på forespørgsel oplyser 84,6\%, at det er fordi, at de ikke kender MinSP. Dette fremgår også af vores interview undersøgelse, hvor 4 ud af 6 ikke kender MinSP.\\
Det er et problem for Region Sjælland, at patienterne ikke bruger MinSP, da det er en målsætning i Region Sjællands It-strategi \footnote{Projektgrundlag, s. 5-6, afsnit 3.1 IT Strategien}, at regionen ønsker fuldt udbytte af sine investeringer. 
Vi valgte derfor, i vores dybdeanalyse, at sætte 'At skabe opmærksomhed omkring MinSP', som anden prioritet i vores prioriteringsliste af fokuspunkter, men vurderede at dette problem lå udenfor MinSP, fordi vi mener, at det er hospitalet selv og sundhedspersonalet, der har opgaven med systematisk at informere om MinSP. Dette kan ske ved f.eks. at udarbejde pjecer og fortælle om MinSP til diabetikerne via mails, eller når de kommer til kontrol på hospitalet. Der kan evt. også tilbydes læring i brugen af MinSP.\\
Imidlertid mener vi også, at en udvidelse af funktionaliteten, forbedring af brugervenligheden og af informationsniveauet i MinSP, vil kunne understøtte en mere udbredt brug af MinSP.\\
Dette sammenholdt med, at vi ud fra vores spørgeskemaundersøgelse kan se, at 57.1\% fornyer deres recept digitalt, gør, at vi har valgt at opprioriteret 'Receptfornyelse' (prioritet 8), til prioritet 1. Dette understøttes også af vores interviewundersøgelse, hvor 4-5 ud af 6 diabetikere gerne vil forny deres recept digital, men bruger andre systemer som sundhed.dk eller deres læges eget system.
\\ \\
Muligheden for at forny recept via MinSP kan kun ske som en skrevet besked til hospitalet, og funktionaliteten 'Receptfornyelse' er samtidig ikke særlig synlig, da den ligger under hovedmenuen 'Meddelelser' og undermenuen 'Skriv til os' og herefter først i en drop-down menu 'vælg emne' finder man 'Receptfornyelse'. '\\
Vi mener derfor, at dette er en årsag til, at patienterne ikke bruger dette modul, men bruger andre digitale sider. Ved at gøre tilgangen til modulet mere synligt, og samtidig videreudvikle det til en funktionalitet, hvor man hurtigere kan forny sine recepter samt have overblik over sin ordinerede medicin, mener vi, at det vil motivere patienterne til at bruge dette modul i MinSP. \\
Dette også sammenholdt med, at vi ud fra vores spørgeskemaundersøgelse kan se, at 21,43\% med sikkerhed, og 42,86\% sandsynligt ville forny deres recept via MinSP, mener vi, at disse ændringer ville kunne få flere til at bruge MinSP. \\
Derved understøttes flere af målsætningerne i Region Sjællands It-startegi \footnote{Projektgrundlag s. 5-6, afsnit 3.1 IT strategien, punkt 2, 6 og 7}.\\
Implementeringen af 'Receptfornyelse' understøtter også målsætningen i Region Sjællands it-infrastruktur om, at "Brugeren skal kunne finde løsninger i samme brugervendte arbejdsgange,..."\footnote{Projektgrundlag s. 7, afsnit 3.4 It-infrastruktur, punkt 3}, da MinSP, med tilføjelse af receptfornyelse, i højere grad bliver en samlet side for alle funktioner. 
\\ \\
Overnævnte argumentation er årsag til, at vi har opprioriteret prioritet 8 om 'Receptfornyelse' til prioritet 1. 
\newpage
\section{Valg af prototype}
Vi vil udarbejde prototype for receptfornyelse, fordi receptfornyelse er den af vores 3 prioriteter hvor flest interessenter (patient, læge, lægesekreter, apotek)  er involveret og har interaktions med funktionen. Man kan derfor med prototypen få afprøvet arbejdsorganiseringen mellem disse. Det også den af vores prioriteter der har mest ny funktionalitet. Der vil også blive afprøvet kommunikation med databaser.