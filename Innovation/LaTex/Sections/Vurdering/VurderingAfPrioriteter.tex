%
% 
% Er i gang med at rette i denne fil i forbindelse med review
%
\section{Vurdering af prioriteter og valg af prototype}
%Efter en vurdering af vores fokuspunkter, fra dybdeanalysen, ud fra MoSCoW-prioritering (Bilag 2), på baggrund af en udarbejdede furture-user stories (Bilag 1), kunne vi i projektgruppen se, at der var brug for en omprioritering af fokuspunkterne fra dybdeanalysen.\\
\subsection{Vurdering af prioritering af visioner}
Som indledelse til fornyelsesfasen udarbejde vi, på baggrund af en række furture-user stories \footnote{Bilag 1}, en MoSCoW-prioritering \footnote{Bilag 2}, til brug for vurdering af vores prioritering af fokuspunkter fra dybdeanalysen.
%som hjælp til vurdering af en prioritering af vores fokuspunkter, fra dybdeanalysen.\\
Resultatet fra MoSCoW prioriteringen viste, at der var behov for en omprioritering i forhold til vores oprindelige prioritering fra dybdeanalysen. \footnote{Dybdeanalysen afsnit 4 Prioriteter s.16-17}
%Dette gav anledning til en omprioritering af den prioritering vi havde fortaget i dybdeanalysen.\\
%
MoSCoW-prioriteringen viser, at det er fokuspunkterne 'Receptfornyelse', 'Samling af al information' og 'Uniforme prøvesvar' der bør have højeste prioritet i det videre arbejde med forbedring af MinSP.\\
\\
I vurderinger af prioritetsrækkefølgen, for de tre kandidater, har vi set på i hvilken grad de opfylder målsætningerne i Region Sjællands it-strategi og it-infrastruktur. \\
Vi har også inddraget resultater fra vores spørgeskema-undersøgelse \footnote{fra forberedelses-fasen og fokuserings-fasen} og vores interview-undersøgelse \footnote{fra fordybelsesfasen}.
Endvidere har vi lagt vægt på at prioritere, de tre kandidater, efter om de kan bidrage til en mere udbredt brug af MinSP, fordi vores brugerundersøgelser, som konkluderet i dybdeanalysen \footnote{Dybdeanalysen s. 1, s. 12, s.16}, viste at mange diabetespatienter ikke kender/bruger MinSP. 
Spørgerskemaundersøgelsen viser, at 58,70\% ikke bruger MinSP \footnote{Projektgrundlag Bilag punkt 1} og på forespørgsel oplyser 84,6\%, at det er fordi, de ikke kender MinSP. \footnote{Projektgrundlag Bilag 1 punkt 15} Interviewundersøgelsen viser, at 4 ud af 6 ikke kender MinSP. \footnote{Dybdeanalyse afsnit 2.1-2.5 s. 2 - s.9}\\
\\
Dette er et problem i forhold til målsætningerne i Region Sjælland IT Strategi om at: 'Regionen vil have fuld udbytte af sine investeringer, og de nye løsninger skal bruges i deres fulde omfang' \footnote{Projektgrundlag, s. 5, afsnit 3.1 IT Strategien punkt 2}\\
\\
Derfor valgte vi i vores oprindelig prioritering i fordybelsefasen, 'At skabe opmærksomhed omkring MinSP' som prioritet 2, men vurderede, at problemet lå udenfor MinSP, fordi vi mener, at det er hospitalet selv og sundhedspersonalet, der har opgaven med systematisk at informere om MinSP. Dette kan ske ved f.eks, at udarbejde pjecer og fortæller om MinSP til diabetikere via mails, eller når de kommer til kontrol på hospitalet. \\
\\
Men imidlertid, mener vi også, at en udvidelse af funktionaliteter, en forbedring af brugervenligheden og af informationsniveauet i MinSP, vil kunne understøtte en mere udbredt brug af MinSP.
\subsubsection{Receptfornyelse}
Muligheden for at forny recept via MinSP kan kun ske som en skrevet besked til hospitalet, og funktionaliteten 'Receptfornyelse' er samtidig ikke særlig synlig, da den ligger under hovedmenu 'Meddelelser' og undermenu 'Skriv til os' og herefter først i drop-drown menu 'vælg emne' finder man 'Receptfornyelse'.\\
\\
Vi mener, at dette kan være en medvirkende årsag til, at patienterne ikke bruger dette modul, men bruger andre digitale sider. Ved at gøre tilgangen til modulet mere synligt, og samtid videreudvikle det til en funktionalitet, hvor man hurtigere kan forny sine recepter samt have overblik over sin ordinerede medicin, mener vi, at dette vil motivere patienterne til at bruge dette modul i MinSP.\\
\\
Vi mener dette fordi, at vi fra vores spørgeskemaundersøgelse kan se, at 57,1\% fornyer deres recept digitalt \footnote{Projektgrundlag Bilag 1 pkt. 2} og fra vores interviewundersøgelse at 4-5 ud af 6 diabetikere gerne vil forny deres recept digitalt, men bruger andre systemer som sundhed.dk eller deres læges eget system. \footnote{Dybdeanalysen afsnit 2.1 - 2.5 s. 2-9} samt fra spørgeskemaundersøgelsen, at 21,43\% med sikkerhed og 42,86\% sandsynligt ville forny deres recept via MinSP, hvis muligt. \footnote{Projektgrundlag Bilag 1 pkt. 3} \\
\\
Dette viser, at der er interesse for at kunne forny sin recept digitalt, og også for at kunne gøre det via MinSP.\\
\\
Samlet set, er der derfor grundlag for, at ændringerne vil kunne motivere flere til at bruge MinSP.\\
\\
En implementering af 'Receptfornyelse' som beskrevet ovenfor vil opfylde flere målsætninger i Region Sjællands IT-Strategi \footnote{Projektgrundlag afsnit 3.1 IT-Strategien punkt 2,6,7 s.5-6} samt målsætningen i Region Sjællands IT-infrastruktur om at: 'Brugeren skal kunne finde løsninger i samme brugervendte arbejdsgange...' \footnote{Projektgrundlag afsnit 3.4 IT-Infrastruktur punkt 3, s.7}\\
\\
I fordybelsesfasen havde vi placeret 'Receptfornyelse' som prioritet 8, men vi mener at ovennævnte argumentation sammenholdt med resultatet fra vores MoSCoW prioritering, begrunder, at det vil være en rigtig beslutning, at opprioritere fokuspunktet 'Receptfornyelse' fra prioritet 8 til prioritet 1.
\subsubsection{Samling af al information}
'Samling af al information' var oprindeligt tænkt som nyudviklet side, hvor forskellige information til patienterne om diabetes, kostvejledning m.v. skulle være tilgængelig. Vi har efterfølgende modificeret til en løsning der anvender en ’standardløsning ’ med links, da en nyudviklet side ville blive en for ressourcekravene løsning for sundhedspersonalet at vedligeholde, og det ville ikke stemme overens med målsætningen i Region Sjællands IT-strategi om at reducere sit ressourceforbrug \footnote{Projektgrundlag afsnit 3.1 IT-strategien pkt. 7, s. 6}.\\
De fleste af vores interviewede personer viste interesse for en infoside\footnote{Dybdeanalyse afsnit 2.1-2.5 s. 2-9}, mens dette dog ikke var tilfældet ved spørgeskemaundersøgelsen. \footnote{Projektgrundlag Bilag 1 pkt. 7}\\
\\
Den modificerede løsning opfylder flere af målsætningerne i Region Sjællands IT-Strategi \footnote{Projektgrundlag afsnit 3.1 IT-strategien pkt. 7, 8  s.5-6} og IT-Infrastruktur\footnote{Projektgrundlag afsnit 3.4 IT-strategien pkt. 3, 4 s. 7}, herunder bl.a.: 
\begin{itemize}
\item Brugerne skal kunne betjene sig selv og opleve en bedre servicegrad samtidig med at regionen reducerer sit ressourceforbrug.
\item Vigtigt med systemer som kan hente information fra andre systemer. 
\item Brugeren skal kunne finde løsninger i samme brugervendte arbejdsgange, uden at brugeren skal opleve behov for at navigere mellem forskellige løsninger. Regionen ønsker systemer, som kan hente information fra andre systemer, således at brugeren kun skal henvende sig et enkelt sted.
\item Integrerede selvbetjeningsløsninger.
\end{itemize}
'Standardløsningen' med links vil være billig at udvikle og implementere, og den vil give patienterne let adgang til nyttig information, når de alligevel er logget på MinSP. Vi mener derfor, at løsningen også vil kunne bidrage til en større interesse for at bruge MinSP.\\
\\
I dybdeanalysen havde vi givet ’Samling af al information’ prioritet 1. Vi vil her omprioritere den til prioritet 2, da vi vurderer at argumentationen for at ’Receptfornyelse ’ skal være prioritet 1 vejer tungere.
\subsubsection{Uniforme Prøvesvar}
’Uniforme prøvesvar’, som omfatter mere information og forklaring til prøvesvar, var også prioriteret højt (lige efter ’Samling af al information’) i dybdeanalysen, så reelt ingen ændring af prioriteringen her.   
%\\\\
%\\\\
%\\\\
%\\\\
%\\\\
%Efter grundig vurdering af vores prioritering af fokuspunkter fra dybdeanalysen, har vi valgt at ændre i denne prioritering og har i stedet, som de tre vigtigste fokuspunkter i vores videre arbejde med forbedring af MinSP. valgt følgende:\\
%1. 'Receptfornyelse'\\ 
%2. 'Samling af al information'\\ 
%3. 'Uniforme prøvesvar'.\\
%\\
%Vores oprindelige prioritet 1 'Samling af al information' har vi ændret til prioritet 2.
%Vores argumentation herfor er, at det ville blive en for dyr løsning for sundhedspersonalet at vedligeholde, hvis vi som vi havde beskrevet, lavede funktionaliteten som en ny-udvikling.  
%Ligeledes viste vores spørgeskemaundersøgelse, at 4 ud af 5 diabetikere, hvor den 5'te var neutral, ikke havde ønske om at MinSP skulle bruges til generel information omkring sygdommen diabetes. Dette er dog i modstrid med vores interview-undersøgelse, hvor dette ønskes af de fleste. Derfor har vi beholdt den som prioritet 2, men modificeret til en løsning, der anvender en 'standardløsning' med links. \\
%Vores oprindelige prioritet 2 'At skabe opmærksomhed omkring MinSP' og prioritet 3 'Overflod af fagtermer' har vi vurderet til at ligge udenfor MinSP. Vores oprindelige prioritet 4 'Uniforme prøvesvar' bliver derfor rykket op til prioritet 3.
%\\ \\
%Det fremgår af Region Sjællands It-strategi \footnote{Projektgrundlag, s. 5, afsnit 3.1 IT Strategien}, at "...nye løsninger skal bruges i deres fulde omfang", og at patienterne og borgere skal opleve en bedre servicegrad, og at de, i så høj grad som muligt, skal kunne betjene sig selv.\\
%Men af vores brugere-undersøgelser fremgår det, fra både spørgeskemaer og interviews, at der ikke er så mange der kender/bruger MinSP. Spørgeskemaerne viser, at 58,70\% ikke bruger MinSP, og på forespørgsel oplyser 84,6\%, at det er fordi, at de ikke kender MinSP. Dette fremgår også af vores interview undersøgelse, hvor 4 ud af 6 ikke kender MinSP.\\
%Det er et problem for Region Sjælland, at patienterne ikke bruger MinSP, da det er en målsætning i Region Sjællands It-strategi \footnote{Projektgrundlag, s. 5-6, afsnit 3.1 IT Strategien}, at regionen ønsker fuldt udbytte af sine investeringer. 
%Vi valgte derfor, i vores dybdeanalyse, at sætte 'At skabe opmærksomhed omkring MinSP', som anden prioritet i vores prioriteringsliste af fokuspunkter, men vurderede at dette problem lå udenfor MinSP, fordi vi mener, at det er hospitalet selv og sundhedspersonalet, der har opgaven med systematisk at informere om MinSP. Dette kan ske ved f.eks. at udarbejde pjecer og fortælle om MinSP til diabetikerne via mails, eller når de kommer til kontrol på hospitalet. Der kan evt. også tilbydes læring i brugen af MinSP.\\
%Imidlertid mener vi også, at en udvidelse af funktionaliteten, forbedring af brugervenligheden og af informationsniveauet i MinSP, vil kunne understøtte en mere udbredt brug af MinSP.\\
%Dette sammenholdt med, at vi ud fra vores spørgeskemaundersøgelse kan se, at 57.1\% fornyer deres recept digitalt, gør, at vi har valgt at opprioriteret 'Receptfornyelse' (prioritet 8), til prioritet 1. Dette understøttes også af vores interviewundersøgelse, hvor 4-5 ud af 6 diabetikere gerne vil forny deres recept digital, men bruger andre systemer som sundhed.dk eller deres læges eget system.
%\\ \\
%Muligheden for at forny recept via MinSP kan kun ske som en skrevet besked til hospitalet, og funktionaliteten 'Receptfornyelse' er samtidig ikke særlig synlig, da den ligger under hovedmenuen 'Meddelelser' og undermenuen 'Skriv til os' og herefter først i en drop-down menu 'vælg emne' finder man 'Receptfornyelse'. \\
%Vi mener derfor, at dette er en årsag til, at patienterne ikke bruger dette modul, men bruger andre digitale sider. Ved at gøre tilgangen til modulet mere synligt, og samtidig videreudvikle det til en funktionalitet, hvor man hurtigere kan forny sine recepter samt have overblik over sin ordinerede medicin, mener vi, at det vil motivere patienterne til at bruge dette modul i MinSP. \\
%Dette også sammenholdt med, at vi ud fra vores spørgeskemaundersøgelse kan se, at 21,43\% med sikkerhed, og 42,86\% sandsynligt ville forny deres recept via MinSP, mener vi, at disse ændringer ville kunne få flere til at bruge MinSP. \\
%Derved understøttes flere af målsætningerne i Region Sjællands It-startegi \footnote{Projektgrundlag s. 5-6, afsnit 3.1 IT strategien, punkt 2, 6 og 7}.\\
%Implementeringen af 'Receptfornyelse' understøtter også målsætningen i Region Sjællands it-infrastruktur om, at "Brugeren skal kunne finde løsninger i samme brugervendte arbejdsgange,..."\footnote{Projektgrundlag s. 7, afsnit 3.4 It-infrastruktur, punkt 3}, da MinSP, med tilføjelse af receptfornyelse, i højere grad bliver en samlet side for alle funktioner. 
%\\ \\
%Overnævnte argumentation er årsag til, at vi har opprioriteret prioritet 8 om 'Receptfornyelse' til prioritet 1. 
\subsection{Valg af prototype}
Projektgruppen har valgt at ville udarbejde prototype receptfornyelse, da det er den funktionalitet blandt vores 3 prioriteter, fra MoSCoW-prioriteringen, der skal ny-udvikles og hvor flest interessenter (patient, læge, lægesekreter, apotek) er involveret og har interaktions med funktionen. Med prototypen vil disse interessenter kunne få visualiseret og afprøvet arbejdsorganisationen imellem sig. ’Receptfornyelse’ er også den funktion, der har mest ny funktionalitet for interessenterne. 
%Man kan derfor med prototypen få afprøvet arbejdsorganiseringen mellem disse. Det også den af vores prioriteter der har mest ny funktionalitet. 
Der vil også blive afprøvet kommunikation med databaser. \\ 
\\
Udviklingen af en prototype vil understøtte MUST-metodens princip om reel bruger deltagelse.