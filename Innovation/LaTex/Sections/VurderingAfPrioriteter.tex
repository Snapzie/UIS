\section{Vurdering af prioriteter og valg af prototype}
Efter en vurdering af vores prioriteter fra dybde analyse har vi som kandidater til forbedring af MinSP valgt følgende fokuspunkter: 'Receptfornyelse', 'Samlign af al information' og 'Uniforme prøvesvar'.
\\\\
Vores oprindelige prioritet 1 'Samling af al information' har vi nedprioriteret til prioritet 2. Vores argumentation herfor er, at det ville blive en for dyr løsning for sundhedspersonalet at vedligeholde. 
\todo{Casper: Hvad tænker du bliver for dyrt? 
	Svar: Jeg argumenterer for dette senere under Arbejdes organisering:
	Hvis en side skal ny udvikles og vedligeholdes, vil dette koste og derfor et det en dyr løsning. 
	Hvis siden skal ny udvikles er der nogen der skal skrive alt information / optage videoer osv. Derudover skal alt dette løbende holdes ajour af sundhedspersonalet. Det er derfor en dyr løsning.
	%
	% Note: skal der argumenteres mere her
} 
Ligeledes viste vores spørgeskemaundersøgelse, at 4 ud af 5 diabetiker, hvor den 5'te var neutral, ikke havde ønske om at MinSP skulle bruges til general information omkring sygdommen diabetes. Dette er dog i modstrid med vores interview-undersøgelse, hvor dette ønskes af de fleste.\\
For prioritet 2 'At skabe opmærksomhed omkring MinSP' og prioritet 3 'Overflod af fagtermer' har vi vurderet at løsningen ligger udenfor MinSP. Vores 3 prioritet bliver derfor 'Uniforme prøvesvar'.
\todo{Casper: Jeg forstår ikke sammenhængen?
	Svar: Vi har sagt at prioritet 2 og 3 ikke kan løses af os - ligger uden for min sundhedsplatformen. Derfor rykkes prioritet 4 op - for denne kan løses af os.
	%
	% Note: skal dette forklares på en anden måde
	% Note: Finn: Evt. brug Diagnostikkort / "måske indrag jeres MosCow Pritoting", og senere Virtuellekort ud fra Diagnostikkort
}
\\\\
Det fremgår af Region Sjællands It-strategi \footnote{Projektgrundlag, s. 5, afsnit 3.1 IT Strategien}, at "...nye løsninger skal bruges i deres fulde omfang" og at patienterne og borgere skal opleve en bedre servicegrad og at de i så høj grad så muligt skal kunne betjene dem selv.\\
Som det fremgår af vores brugere-undersøgelser, fra både spørgeskemaer og interviews, er der ikke så mange der kender/bruger MinSP. Spørgeskemaerne viser, at 58,70\% ikke bruger MinSP, og på forespørgsel oplyser 84,6\%, at det er fordi, at de ikke kender MinSP. Dette fremgår også af vores interview undersøgelse, hvor 4 ud af 6 ikke kender MinSP.\\
Det er et problem for Region Sjælland, at patienterne ikke bruger MinSP, da det er en målsætning i Region Sjællands It-strategi \footnote{Projektgrundlag, s. 5-6, afsnit 3.1 IT Strategien}, at regionen ønsker fuldt udbytte af sine investeringer. 
Vi valgte derfor, at sætte 'At skabe opmærksomhed omkring MinSP', som anden prioritet i vores prioriteringsliste af fokuspunkter, men vurderet at dette problem lå udenfor MinSP. Det gjorder vi, fordi vi mener, at det er hospitalet selv og sundhedspersonalet, der har opgaven med systematisk at informere om MinSP. Dette kan ske ved f.eks. at udarbejde pjecer og fortælle om MinSP til diabetikerne via mails, eller når de kommer til kontrol på hospitalet. Der kan evt. også tilbydes læring i brugen af MinSP.\\
Imidlertid mener vi også, at en udvidelse af funktionaliteten, forbedring af brugervenligheden og af informationsniveauet, vil kunne understøtte en mere udbredt brug af MinSP.\\
Dette sammenholdt med, at vi ud fra vores spørgeskema undersøgelse kan se, at 57.1\% fornyer deres recept digitalt, gør, at vi har valgt at opprioriteret 'Receptfornyelse' (prioritet 8), til prioritet 1. Dette understøttes også af vores interview undersøgelse, hvor 4-5 ud af 6 diabetikere gerne vil forny deres recept digital, men bruger andre systemer som sundhed.dk eller deres læges eget system.
\\\\
Muligheden for at forny recept via MinSP kan kun ske som en skrevet besked til hospitalet, og funktionaliteten 'Receptfornyelse' er samtidig ikke særlig synlig, da den ligger under hovedmenuen 'Meddelelser' og undermenuen 'Skriv til os' og herefter først i en drop-down menu 'vælg emne' findes 'Receptfornyelse'. '\\
Vi mener derfor, at dette også er en årsag til, at patienterne ikke bruger dette modul, men bruger andre digitale sider. Ved at gøre tilgangen til modulet mere synligt, og samtidig videreudvikle det til en funktionalitet, hvor man hurtigere kan forny sine recepter samt have overblik over sin ordinerede medicin, mener vi, at det vil motivere patienterne til at bruge dette modul i MinSP. Dette også sammenholdt med, at vi udefra vores spørgeskemaundersøgelse kan se, at 21,43\% med sikkerhed, og 42,86\% sandsynligt ville forny deres recept via MinSP, mener vi, at disse ændringer ville kunne få flere til at bruge MinSP. Derved understøttes flere af målsætningerne i Region Sjællands It-startegi \footnote{Projektgrundlag s. 5-6, afsnit 3.1 IT strategien, punkt 2, 6 og 7}. Implementeringen af 'Receptfornyelse' understøtter også målsætningen i Region Sjællands it-infrastruktur om, at "Brugeren skal kunne finde løsninger i samme brugervendte arbejdsgange, uden at brugerne skal opleve behov for at navigere mellem forskellige løsninger"\footnote{Projektgrundlag s. 7, afsnit 3.4 It-infrastruktur, punkt 3}, da MinSP i højere grad bliver en samlet side for alle funktioner. 
\\ \\
Overnævnte argumentation er årsag til, at vi har opprioriteret prioritet 8 om 'Receptfornyelse' til prioritet 1. 
\\\\
Vi har derfor valgt 'Receptfornyelse' som den vigtigste forbedring og valgt at udarbejde en protype for denne funktionalitet.