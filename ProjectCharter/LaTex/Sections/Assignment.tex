\subsection{Opgave og formål}
Projektets overordnede formål vil være, at i samarbejde med sundhedspersonale, diabetikere og region hovedstaden at realisere visionen om forbedringer til 'Min Sundhedsplatform'. Ved projektets begyndelse fandt vi følgende cases som vi mente ville være forbedringer til 'Min Sundhedsplatform'.
\subsubsection{Receptfornyelse}
Vi mente at netop det at skulle fornye sin recept var en enkel process som i forvejen kan gøres online men på sundhed.dk. Ved at gøre det muligt at fornye sin recept på Min Sundhedsplatform ville man samle flere (hvis ikke alle) arbejdsprocessor til den samme side for diabetikere, og det vil derfor ikke kræve at diabetikere skal sætte sig ind i flere sider for at få behandling.
\subsubsection{Læring og videnscenter}
Vi identificerede at der ikke var meget almen information omkring diabetes og patienters sygdom generelt på Min Sundhedsplatform. Vi ville derfor gerne kigge videre på mulighederne i at samle viden omkring sygdommen, såsom hvad man skal gøre hvis man oplever situation x eller y. Ved at samle information og viden på Min Sundhedsplatform vil man give diabetikeren et enkelt sted hvor de burde kunne finde al den nødvendige information. Dette videnscenter vil hovedsagligt henvende sig til den nye eller uerfarne diabetiker som måske lige er blevet oplært i sin sygdom. Denne udvidelse af Min Sundhedsplatform ville også kunne tjene til motivation frem mod diabetikerens nye livsstil.
\subsubsection{Socialt samlingssted}
Ved at inkorporere et socialt samlingssted på Min Sundhedsplatform ville man kunne forbinde mennesker med andre som deler de samme bekymringer ved at have fået konstateret diabetes. Den sociale del kunne bestå af både større grupper hvor man kan dele viden og erfaringer eller det kunne bestå af mindre men mere intime grupper som måske ønsker at mødes og være sammen. Forumet ville lægge op til at dele viden og erfaring, og motivere hinanden. Man kunne forstille sig at en lille gruppe ældre kunne finde sammen nogle gange om ugen og cykle en tur og på den måde motivere hinanden til den ændrede livsstil.
\subsubsection{Forberedelse af fremtiden}
I fremtiden vil vi helt sikkert se teknologisk fremgang i forhold til behandlingen af diabetes. Vi forestiller os at et fremtidigt initiativ kunne være at implementere en chip hos patienterne som hele tiden måler blodsukkeret. Denne case ville fokusere på hvordan man i fremtiden ville kunne benytte sig af det potentiale som det ville være hvis diabetikerens blodsukker hele tiden blev målt, og hvordan det potentiale kunne bruges af Min Sundhedsplatform. Man kunne forestille sig at ved hjælp af machine learning og databehandling kunne man forudse unormale ændringer i blodsukkeret og advare diabetikeren, for eksempel gennem en app, og relevant sundhedspersonale.\\\\
Blandt de fire cases valgte vi at skrive et spørgeskema som gav os mulighed for at se om der er nogen relevans for vores forslag. Gennem princippet om brugerindragelse kom vi frem til at arbejde videre med \todo{Videre arbejde} 