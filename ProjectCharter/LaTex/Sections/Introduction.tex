\section{Baggrund}
Sundhedsplatformen blev valgt af regionerne i november 2013 og er blevet udviklet af det amerikanske selskab Epic. Sundhedsplatformen agerer bindeled mellem mange undersystemer og samler alle de funktioner, som de 44000 danske sundhedsprofessionelle benytter mest. I Danmark bliver Sundhedsplatformen benyttet på 17 sygehuse eller hospitaler til 2.5 millioner borgere i Region Sjælland og Region Hovedstaden. På verdensplan er Sundhedsplatformen i drift på 1100 hospitaler og omfatter 172 millioner patienter.\\
Patienter har adgang til Sundhedsplatformen gennem 'Min Sundhedsplatform' (Min SP) som vores projekt vil fokusere på. Gennem Min SP kan patienter skrive beskeder til ambulatorier, anmode om at aflyse og få nye tider, se prøvesvar og få adgang til dele af ens journaler. Min SP og Sundhedsplatformen er stadig under udvikling og læger, sygeplejersker, it-specialister og projektledere i Region Sjælland og Hovedstaden tilpasser og opdatere løbende systemerne til fremtidens behov for behandlinger.