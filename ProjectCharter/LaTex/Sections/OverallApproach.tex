\subsection{Overordnet tilgang}
Tilgangen til projektet er baseret på MUST principperne beskrevet i 'Participatory IT Designing for Business and Workplace Realities'. Ligeledes er de benyttede og de fremtidige redskaber og aktiviteter beskrevet i projektgrundlaget baseret på anbefalinger fra samme bog.\\
Nedenfor findes vores udarbejdede baseline som viser deadlines og produkter for vores projekt. Initiation phase og in-line analysis fasen er blevet lagt sammen, og de aktiviteter beskrevet i denne kombi-fase er blevet udført. Aktiviteterne i de fremtidige faser er foreløbige og stadig til debat hvorvidt de skal ændres. \todo{Ændr engelske begreber?}
\begin{figure}[h!]
	\missingfigure{Billede af Baseline}
	\caption{Udarbejdet baseline}
\end{figure}
Region Sjælland har et strategisk ønske om it systemet skal være gennemtestet samt at opnå fuld tilgængelighed blandt læger og patienter. Af denne årsag ønsker vi at lave en markedsundersøgelse for at komme på sporet. Hertil vil vi benytte os af SWOT-analysen til at visualisere hvorledes analysens elementer vægtes.
Som led i regionens strategi om at klinikere skal have overblik over relevante data og billeder, vil vi arrangere insitu-interview, hvor vi kortlægger hvilke data der er relevante. Insitu-interview vil ligeledes også være relevante ift. patienter og behandlere, for at vi kan nedbringe konsultation- og behandlingstid.   